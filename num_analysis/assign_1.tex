\documentclass[12pt]{report}
\usepackage{fancyhdr}
\usepackage{amsmath}
\usepackage{amssymb}
\usepackage{amsfonts}
\usepackage{amsthm}
\usepackage[a4paper, width=150mm,top=25mm,bottom=25mm]{geometry}
\pagestyle{fancy}
\setlength{\headheight}{14.49998pt}
\fancyhead[L]{\small\leftmark}
\fancyhead[R]{\small\rightmark}
% \theoremstyle{definition}
\newtheorem{thm}{Theorem}
\newtheorem{lem}{Lemma}
\newtheorem{defn}{Definition}
\newtheorem*{rem}{Remark}
\newtheorem{prop}{Proposition}
\newtheorem{proper}{Property}
% Use \paragraph*{} for Ques. and Ans.

\title{
\author{Devansh Tripathi\\ IMS22090\\ Lecturer: Dr. Asha K. Dond}
}

\begin{document}
\maketitle
\section*{Assignment 1}
\paragraph*{Ans 1.} Given $\epsilon = 10^{-2}$\\
\begin{minipage}{0.5\textwidth}
    \begin{align*}
        |a_n| &< 10^{-2}\\
        |a_n| &< \frac{1}{100}\\
        \left|\frac{1}{n+2}\right| &< \frac{1}{100}
    \end{align*}
    $n\in\mathbb{N}$, so $n+2$ is positive for all $n$.
    \begin{align*}
        n+2 &> 100\\
        n &> 98\\
        n &= 99 ~~\text{(least positive integer.)}
    \end{align*} 
\end{minipage}%
\begin{minipage}{0.5\textwidth}
    \begin{align*}
        |b_n| &< 10^{-2}\\
        \left|\frac{1}{n^2}\right| &< \frac{1}{100}
    \end{align*}
    $n\in \mathbb{N}$ and $n^2$ is positive for all $n$.
    \begin{align*}
        n^2 &> 100\\
        n &> 10\\
        n &= 11 ~~\text{(least positive integer.)}
    \end{align*}
\end{minipage}
\paragraph*{Ans 3.} For domain $[0,1]$, the range of the function $f(x) = sin(x) + x^2 -1$ is
\begin{align*}
    R &= sin((0,1)) + ((0,1))^2 - 1\\
    R &= (0, sin(1)) + (0,1) -1 \\
    R &= (0, sin(1)) + (-1, 0)\\
    R &= (-1, sin(1))
\end{align*}
Since, in the interval $(0,1)$ the function is taking the value $(-1, sin(1))$. It is going from negative to positive in $y$ hence it will definately cut the $x$ axis atleast once which will be the root.
\paragraph*{Ans 4.} For $f(x) - x = 0 \label{1}$ in the interval $[0,1]$, if we take $ x= 0$, it is simple to observe that $ f(0) - 0 = 0$. Hence, $x= 0$ is the solution of the given equation \ref{1}in the interval $[0,1]$. $f(0) = 0$ also means that $0$ is the root of $f(x)$. 

\paragraph*{Ans 6.} Taylor's expansion of $f(x)$ at $x_0$ is
$$ f(x) =  \sum\limits_{k=0}^{n} \frac{f^{(k)}(x_0) (x - x_0)^k}{k!} 
$$ where $k \in \mathbb{Z^+}$. For $x_0 = 0$ and $n=3$, we can write
\begin{align*}
    sin(x) &= sin(0) + cos(0)x - sin(0) \frac{x^2}{2!} - cos(0)\frac{x^3}{3!}\\
    sin(x) &= x - \frac{x^3}{6}
\end{align*}
Error term will be 
\begin{align*}
    E &= \frac{x^4}{24}sin(\xi(x))
\end{align*}
where $0 < \xi(x) < x$, for $n = 8$ we have
\begin{align*}
    sin(x) &= sin(0) + cos(0)x - sin(0) \frac{x^2}{2!} - cos(0)\frac{x^3}{3!} + sin(0) \frac{x^4}{24} + cos(0)\frac{x^5}{120} \\ &- sin(0)\frac{x^6}{720} - cos(0)\frac{x^7}{7!} + sin(0)\frac{x^8}{8!}\\
    sin(x) &= x - \frac{x^3}{6} + \frac{x^5}{120} - \frac{x^7}{7!}
\end{align*}
Error term will be
\begin{align*}
    E &= \frac{x^9}{9!}cos(\xi(x))
\end{align*}
where $0 < \xi(x) < x$
\paragraph*{Ans 7.} For $f(x) = cos(x)$ and $n = 3$, Taylor's expansion at $x_0 = 0$ will look like
\begin{align*}
    cos(x) &= cos(0) - sin(0)x - cos(0)\frac{x^2}{2} + sin(0)\frac{x^3}{6}\\
    cos(x) &= 1 - \frac{x^2}{2}
\end{align*}
\begin{align*}
    \int_{0}^{0.4} f(x)dx &\approx \int_{0}^{0.4}\left(1 - \frac{x^2}{2}\right) dx\\
    &\approx \left[x - \frac{x^3}{6}\right]_{0}^{0.4}\\
    & \approx 0.4 - \frac{(0.4)^3}{6}\\
    & \approx 0.389333 \dots
\end{align*}
(b) Error function will look like for $0 < cos(\xi(x)) \leq 1$
\begin{align*}
    E &= \left|cos(\xi(x))\frac{x^4}{24}\right| \leq \left|\frac{x^4}{24}\right|
\end{align*} 
Hence, the upper bound of the error is $\frac{x^4}{24}$.

\paragraph*{Ans 8.} Taylor's expansion of $e^x$ around $x_0 = 0$ with polynomial degree $n = 4$ will be
$$ e^x = 1 + x + \frac{x^2}{2} + \frac{x^3}{6} + \frac{x^4}{24} $$
Putting $x = 1$
\begin{align*}
    e &= 1 + 1 + \frac{1}{2} + \frac{1}{6} + \frac{1}{24} \\
    e &= 2.70833
\end{align*}
Error term will be
$$ error = \left|\frac{e^{\xi(x)} x^{n+1}}{(n+1)!}\right| \leq 10^{-6}$$
Putting $x = 1$
$$ error = \left|\frac{e^{\xi(1)}}{(n+1)!}\right| $$
Since $0< \xi(x) < x = 1$ and $e^x$ is the increasing function, we have $ 0 < e^{\xi(x)} < e$ 
\begin{align*}
    \left|\frac{e^{\xi(1)}}{(n+1)!}\right| &\leq \left| \frac{e}{(n+1)!}\right| \leq 10^{-6}\\
    (n+1)! &\geq \frac{e}{10^{-6}}\\
    (n+1)! &\geq e \times 10^6
\end{align*}
\paragraph*{Ans 9.} Second degree Taylor's approximation for $f(x) = \sqrt{x+1}$ at $x_0 = 0$ is 
\begin{align*}
    \sqrt{x+1} &= 1 + \frac{1}{2\sqrt{x+1}} + \frac{-1}{4(x+1)^{3/2}}\\
    &= 1 + \frac{1}{2\sqrt{2}} - \frac{x^2}{8}
\end{align*}
Remainder term is 
$$ R = \frac{x^3}{16(\xi(x)+1)^{5/2}}$$
where $x_0 = 0 < \xi(x) < x$.
We can obtain a bound for this term if we choose $x \in [0, \infty]$.

\paragraph*{Ans 10.} For $f(x) = \frac{1}{x}$, we have $f'(x) \frac{-1}{x^2}$, $f''(x) = \frac{2}{x^3}$ and $f'''(x) = \frac{-6}{x^4}$\\
General expression for derivative will be
\begin{align*}
    f^{(k)}(x) = \frac{(-1)^k k!}{x^{k+1}}
\end{align*}
General expression for Taylor's expansion will be
\begin{align*}
    \frac{1}{x} &= \sum\limits_{k=0}^{n}\frac{(-1)^k k!(x-x_0)^k}{x_0^{k+1}k!}\\
    P_n(x) = \frac{1}{x} &= \sum\limits_{k=0}^{n}(-1)^k (x-1)^k~~~~ (x_0 = 1)
\end{align*}
$f(3)$ using $P_0(x)$
$$ f(3) = 1$$
$f(3)$ using $P_1(x)$
$$ f(3) = 1 - (3 - 1) = -1$$
$f(3)$ using $P_2(x)$
$$ f(3) = 1 - 2 + (3 - 1)^2 = 3$$
$f(3)$ using $P_3(x)$
$$ f(3) = 1-2+4-(3-1)^3 = -5$$
$f(3)$ using $P_4(x)$
$$ f(3) = 1-2+4-8+(3-1)^4 = 11$$
$f(3)$ using $P_5(x)$
$$ f(3) = 1-2+4-8+16-(3-1)^5 = -21$$
We can observe that the values of $f(3)$ using Taylor's expansion is not converging rather it is oscillating. 
\end{document}
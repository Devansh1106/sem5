\documentclass[12pt]{report}
\usepackage{fancyhdr}
\usepackage{amsmath}
\usepackage{amssymb}
\usepackage{amsfonts}
\usepackage{amsthm}
\usepackage[a4paper, width=150mm,top=25mm,bottom=25mm]{geometry}
\pagestyle{fancy}
\setlength{\headheight}{14.49998pt}
\fancyhead[L]{\small\leftmark}
\fancyhead[R]{\small\rightmark}
% \theoremstyle{definition}
\newtheorem{thm}{Theorem}
\newtheorem{lem}{Lemma}
\newtheorem{defn}{Definition}
\newtheorem*{rem}{Remark}
\newtheorem{prop}{Proposition}

\begin{document}
\section*{Lecture 1: Theory of Groups and Rings}
\begin{defn}
    Fibre of $f$ over $b$: For a function $f:A\to B$, the pre-image of $b \in B$ is called the fibre of $f$ over $b$.
\end{defn}
\begin{defn}
    Equivalence class of $a \in A$ is defined to be $\{x \mid xRa\}$. The elements of equivalence class of $a \in A$ are said to be equivalent to $a$ and any element of this class is called the representative of this class. They are denoted by $[a]$. 
\end{defn}
\begin{lem}
    Any two equivalence classes are either disjoint or equal.
\end{lem}
\begin{proof}
    Suppose, we have two equivalence classes $[a]$ and $[b]$ such that $[a] \neq [b]$. We need to prove that $[a] \cap [b]= \phi$ Suppose for contradiction that $\exists x$ such that $x = [a] \cap [b]$. This means that $xRa$ and $xRb$. Using symmetry of equivalence relation, we have $aRx$ and $xRb$ and by transitivity we can say $aRb$. Hence, $[a] = [b]$ which is a contradiction.\\
    We have prove that if they are not equal then they are disjoint. Other way can be proved with similar argument. 
\end{proof}
\begin{defn}
    Partition of $A$: A partition of $A$ is any collection $\{A_i \mid i \in I\}$ of non-empty subsets of $A$ such that it follows:
    \begin{enumerate}
        \item $\bigcup\limits_{i \in I} A_i = A$, and
        \item $A_i \bigcap A_j = \phi ~~\forall i,j \in I$ and $i \neq j$.
    \end{enumerate}
\end{defn}
\begin{rem}
    The notion of an equivalence relation on $A$ and a partition of $A$ are the same.
\end{rem}
For a set $A$, every equivalence relation on $A$ \textit{induces} the partition on set $A$ using equivalence classes. In other words, every equivalence class associated with a equivalence relation forms partition of $A$.\\
If $R$ is the equivalence relation on $A$ then the induced partition $P$ will be $$ P = \{\{b \mid bRa, \forall b \in A\} \mid a \in A\}$$
Also, with the given partition $P$, we can define relation $R$ as $$ R = \{bRa \mid \exists p_i \in P ~\text{such that}~ a, b \in p_i ~\forall i \in I\}$$
We can say a partition $P$ is made up of equivalence classes $p_i$.
\begin{prop}
    Let $A$ be a nonempty set.
    \begin{enumerate}
        \item If $R$ defines a equivalence relation on $A$ then the set of equivalence classes of $R$ forms a partition of $A$. 
        \item If $\{A_i \mid i \in I\}$ is a partition of $A$ then there is an equivalence relation on $A$ whose equivalence classes are sets $A_i, i \in I$. 
    \end{enumerate}
\end{prop}
\begin{proof}
    \begin{enumerate}
        \item Suppose $P$ is a set of equivalence classes of $R$, defined as $$ P = \{\{b \mid bRa, b \in A\} \mid a \in  A\}$$ We need to prove that $P$ defines partition of $A$. For some $p_i \in P$, it will be nonempty since it will have atleast $a$ which is related to itself. Now, using lemma $1$, we can say that $p_i \cap p_j = \phi$ for some $p_i,p_j \in P~\text{and}~ i,j \in \mathbb{N}$ given that $i \neq j$. Union property TBD.
        \item Given the collection of sets, $Q = \{A_i \mid i \in I\}$ as a partition of $A$, we need to show that there exists an equivalence relation on $A$ with equivalence classes as sets $A_i, i \in I$.\\
        We can define relation $R$ as
        $$ R = \{(a,b) \mid bRa~\text{and}~\exists A_i \in Q~\text{such that}~a,b \in A_i\}$$
        It is reflexive (obvious), symmetric (obvious) and transitive ($aRb$ such that $a,b \in A_i$ and $bRc$ such that $b,c \in A_i \implies aRc$ such that $a,c \in A_i$). Hence, it is a equivalence relation and the corresponding equivalence class for some $i$ will be 
        $$ q_i = \{b \mid bRa_i, \forall b \in A~ \&~ a_i \in A\}$$ 
        Since with the same arguments given in proof of part $(1)$ that $q_i, i \in I$ is nonempty, disjoint and exhausts the set $A$, we can say it forms the partition of $A$ and since every equivalence relation induces a unique partition, we can also say that $q_i, i \in I$ are precisely the sets $A_i, i \in I$.
    \end{enumerate}
\end{proof}




\end{document}

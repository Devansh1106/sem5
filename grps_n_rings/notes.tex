\documentclass[12pt]{report}
\usepackage{fancyhdr}
\usepackage{amsmath}
\usepackage{amssymb}
\usepackage{amsfonts}
\usepackage{amsthm}
\usepackage[a4paper, width=150mm,top=25mm,bottom=25mm]{geometry}
\pagestyle{fancy}
\setlength{\headheight}{14.49998pt}
\fancyhead[L]{\small\leftmark}
\fancyhead[R]{\small\rightmark}
% \theoremstyle{definition}
\newtheorem{thm}{Theorem}
\newtheorem{lem}{Lemma}
\newtheorem{defn}{Definition}
\newtheorem*{rem}{Remark}
\newtheorem{prop}{Proposition}
\newtheorem{proper}{Property}

\title{
\author{Devansh Tripathi\\ Lecturer:}
}

\begin{document}
\maketitle
\section*{Lecture 1: Theory of Groups and Rings}
\begin{defn}
    Fibre of $f$ over $b$: For a function $f:A\to B$, the pre-image of $b \in B$ is called the fibre of $f$ over $b$.
\end{defn}
\begin{defn}
    Equivalence class of $a \in A$ is defined to be $\{x \mid xRa\}$. The elements of equivalence class of $a \in A$ are said to be equivalent to $a$ and any element of this class is called the representative of this class. They are denoted by $[a]$. 
\end{defn}
\begin{lem}
    Any two equivalence classes are either disjoint or equal.
\end{lem}
\begin{proof}
    Suppose, we have two equivalence classes $[a]$ and $[b]$ such that $[a] \neq [b]$. We need to prove that $[a] \cap [b]= \phi$ Suppose for contradiction that $\exists x$ such that $x = [a] \cap [b]$. This means that $xRa$ and $xRb$. Using symmetry of equivalence relation, we have $aRx$ and $xRb$ and by transitivity we can say $aRb$. Hence, $[a] = [b]$ which is a contradiction.\\
    We have prove that if they are not equal then they are disjoint. Other way can be proved with similar argument. 
\end{proof}
\begin{defn}
    Partition of $A$: A partition of $A$ is any collection $\{A_i \mid i \in I\}$ of non-empty subsets of $A$ such that it follows:
    \begin{enumerate}
        \item $\bigcup\limits_{i \in I} A_i = A$, and
        \item $A_i \bigcap A_j = \phi ~~\forall i,j \in I$ and $i \neq j$.
    \end{enumerate}
\end{defn}
\begin{rem}
    The notion of an equivalence relation on $A$ and a partition of $A$ are the same.
\end{rem}
For a set $A$, every equivalence relation on $A$ \textit{induces} the partition on set $A$ using equivalence classes. In other words, every equivalence class associated with a equivalence relation forms partition of $A$.\\
If $R$ is the equivalence relation on $A$ then the induced partition $P$ will be $$ P = \{\{b \mid bRa, \forall b \in A\} \mid a \in A\}$$
Also, with the given partition $P$, we can define relation $R$ as $$ R = \{bRa \mid \exists p_i \in P ~\text{such that}~ a, b \in p_i ~\forall i \in I\}$$
We can say a partition $P$ is made up of equivalence classes $p_i$.
\begin{prop}
    Let $A$ be a nonempty set.
    \begin{enumerate}
        \item If $R$ defines a equivalence relation on $A$ then the set of equivalence classes of $R$ forms a partition of $A$. 
        \item If $\{A_i \mid i \in I\}$ is a partition of $A$ then there is an equivalence relation on $A$ whose equivalence classes are sets $A_i, i \in I$. 
    \end{enumerate}
\end{prop}
\begin{proof}
    \begin{enumerate}
        \item Suppose $P$ is a set of equivalence classes of $R$, defined as $$ P = \{\{b \mid bRa, b \in A\} \mid a \in  A\}$$ We need to prove that $P$ defines partition of $A$. For some $p_i \in P$, it will be nonempty since it will have atleast $a$ which is related to itself. Now, using lemma $1$, we can say that $p_i \cap p_j = \phi$ for some $p_i,p_j \in P~\text{and}~ i,j \in \mathbb{N}$ given that $i \neq j$. \\
        Also, we know that every point of set $A$ will be in some equivalence class (reflexivity so atleast in the equivalence of itself). If we take union of all those classes we will get $A$ as each point has a atleast a equivalence class.
        \item Given the collection of sets, $Q = \{A_i \mid i \in I\}$ as a partition of $A$, we need to show that there exists an equivalence relation on $A$ with equivalence classes as sets $A_i, i \in I$.\\
        We can define relation $R$ as
        $$ R = \{(a,b) \mid bRa~\text{and}~\exists A_i \in Q~\text{such that}~a,b \in A_i\}$$
        It is reflexive (obvious), symmetric (obvious) and transitive ($aRb$ such that $a,b \in A_i$ and $bRc$ such that $b,c \in A_i \implies aRc$ such that $a,c \in A_i$). Hence, it is a equivalence relation and the corresponding equivalence class for some $i$ will be 
        $$ q_i = \{b \mid bRa_i, \forall b \in A~ \&~ a_i \in A\}$$ 
        Since with the same arguments given in proof of part $(1)$ that $q_i, i \in I$ is nonempty, disjoint and exhausts the set $A$, we can say it forms the partition of $A$ and since every equivalence relation induces a unique partition, we can also say that $q_i, i \in I$ are precisely the sets $A_i, i \in I$.
    \end{enumerate}
\end{proof}
\subsection*{Properties of Integers}
\begin{proper}
    Well Ordering of $\mathbb{Z}$: If $A$ is any nonempty subset of $\mathbb{Z^+}$, there is some element $m \in A$ such that $m \leq a$, $\forall a \in A$. $m$ is called the minimum element of $A$. 
\end{proper}
\begin{proof}
    Proof for well ordering property of $\mathbb{Z}$ i.e. existence and uniqueness of minimal element.\\
    For a nonempty subset $A$ of $\mathbb{Z^+}$, we can prove this by induction. Suppose $A = \{a_1\}$ then $a_1 \leq a_1$ hence $a_1$ is the minimal element. Suppose there exists a minimal element $m$ in the set $A$ have $n$ element $a_1, a_2 ,~\dots~ a_n$. \\
    For the case when $A = \{a_1, a_2, ~\dots~a_{n+1}\}$, we already have $m$ as the minimal element for $\{a_1, a_2,~\dots~a_n\}$ hence there are three cases: if $a_{n+1} = m$, $a_{n+1} > m$ and $a_{n+1} < m$ and in all three cases there exists a minimal element in set $A$.\\
    For uniqueness of minimal element, suppose for contradiction there exists two minimal elements in set $A$, say $m_1, m_2$ such that $m_1 \neq m_2$. Then there are two possibilities $m_1 > m_2$ and $m_1 < m_2$. If $m_1 > m_2$ then assume $m_1$ to be the minimal element hence this case is not possible. Similarly, $m_1 < m_2$ is also not possible assuming $m_2$ be the minimal element. Hence, contradiction. Therefore, $m_1 = m_2$. 
\end{proof}
\begin{defn}
    $ a | b =$ $a$ divides $b$. $b = ac$ for some $c \in \mathbb{Z}$.
\end{defn}
\begin{defn}
    For some $a, b \in \mathbb{Z}$, denote $d = gdc(a,b)$ and $l = lcm(a,b)$ then $dl = ab$.
\end{defn}
\begin{defn}
    The Division Algorithm: If $a,b \in \mathbb{Z} \backslash \{0\}$, then there exists a unique $q,r \in \mathbb{Z}$ such that $a = bq +r$ where $0 \leq r \leq |b|$. And $q$ is called quotient and $r$ is called remainder.
\end{defn}
\begin{defn}
    The Euclidean Algorithm: Suppose $a,b \in \mathbb{Z}\backslash\{0\}$, we can use this algorithm to find the gcd of these two number in the following way:
    \begin{align*}
        a &= q_0b + r_0\\
        b &= q_1r_0 + r_1\\
        r_0 &= q_2r_1 + r_2\\
        \vdots \\
        r_{n-2} &= q_{n+1}r_{n-1} + r_{n}\\
        r_{n-1} &= q_{n+2}r_{n} \\
    \end{align*}
    where $r_n$ is the gcd of $(a,b)$. Such an $r_n$ exists because $|b| > |r_0| > |r_1| ~\dots $ is a decreasing sequence of strictly postive integers hence it cannot go on to infinite elements.
\end{defn}
$\mathbb{Z}$-linear combination of $a$ and $b$: For $a,b \in \mathbb{Z}\backslash\{0\}$, we have $x,y\in \mathbb{Z}$ such that we can write $gcd(a,b)$ as linear combination of $x,y$ $$ gcd(a,b) = ax + by$$. 
\begin{thm}
    Fundamental Theorem of Arithmetic: If $n \in \mathbb{Z}$, $n > 1$, then $n$ can be factored uniquely into the product of primes, i.e. there are distinct primes $p_1, p_2, ~\dots,p_s$ and positive integers $\alpha_1, \alpha_2, ~\dots, \alpha_s$, such that
    $$ n = p_1^{\alpha_1} p_2^{\alpha_2} \dots p_s^{\alpha_s} $$ 
    This factorization is unique in the sense that the set of $p_i's$ is unique and no other set of primes and the exponent can generate the same number.
\end{thm}
\begin{proof}
    We will use induction to prove the first part that every $n > 1$ can be written as the product of primes.\\
    For $n=2$ it is true as $ 2 = 2^1$. Suppose for all the numbers less than $n$ can be written as the product of primes. Now, for $n$ we can have two cases:
    \paragraph{Case: 1} If $n$ is prime then it is obvious that it's true.
    \paragraph{Case: 2} If $n$ is composite then $n$ can be written as $n = ab$ where $ 0 < a,b < n$ by definition of composite numbers. And by our assumption for induction it is true thatn $a,b$ can be written as product of primes as they are less than $n$. Hence, $n = ab$ can also be a product of primes. \\
    Now, about the uniqueness of the primes factors. Let's assume that there exists some primes $q_i's$ and the exponents $\beta_i's$ such that
    $$ n = p_1^{\alpha_1} p_2^{\alpha_2} \dots p_s^{\alpha_s} = q_1^{\beta_1} q_2^{\beta_2} \dots q_s^{\beta_s} $$
    Since $p_1$ divides the left side, it should also divides the right side. Hence, $p_1 | q_i$ for some $i$. But $p_1$ and $q_i$ are primes $\implies$ $p_1 = q_i$. WLOG, we can choose $i = 1$ $\implies$ $p_1 = q_1$.
    $$ p_1^{\alpha_1} p_2^{\alpha_2} \dots p_s^{\alpha_s} = p_1^{\beta_1} q_2^{\beta_2} \dots q_s^{\beta_s}$$
    Now, we can have $ \alpha_1 > \beta_1$ and we can cancel $p_1^{\beta_1}$ from both sides. 
    $$ p_1^{\alpha_1 - \beta_1} p_2^{\alpha_2} \dots p_s^{\alpha_s} = q_2^{\beta_2} \dots q_s^{\beta_s}$$
    But observe that now $p_1$ divides left side but not the right side. Hence $ \alpha_1 \ngtr \beta_1$. Similar argument for $ \alpha_1 < \beta_1$. 
    Therefore, $\alpha_1 = \beta_1$.\\
    Using induction we can show that both sides are equivalent. Hence, primes and their coefficients are unique.
\end{proof}
We can also define lcm and gcd using fundamental theorem of arithmetic as:
\begin{align*}
    gcd(a,b) &= p_1^{\min(\alpha_1,\beta_1)} p_2^{\min(\alpha_2,\beta_2)} ~\dots \\
    lcm(a,b) &= p_1^{\max(\alpha_1,\beta_1)} p_2^{\max(\alpha_2,\beta_2)} ~\dots    
\end{align*}
\begin{defn}
    Euler $\phi$- function: For $n \in \mathbb{Z^+}$ let $\phi(n)$ be the number of positive integers $a \leq n$ with $(a,n) = 1$. For primes $p$, $\phi(p) = p-1$, and more generally, $\forall a \geq 1$ we have
    $$ \phi(p^a) = p^a - p^{a-1} = p^{a-1}(p-1)$$
    The function $\phi$ is {\em multiplicative} in the sense that $ \phi(ab) = \phi(a)\phi(b)$  if $(a,b) = 1$. So for some $n = p_1{\alpha_1}p_2{\alpha_2}~\dots ~p_s{\alpha_s}$ we can write
    \begin{align*}
        \phi(n) &= \phi(p_1^{\alpha_1}p_2^{\alpha_2}~\dots ~p_s^{\alpha_s}) = \phi(p_1^{\alpha_1})\phi(p_2^{\alpha_2}) ~\dots~ \phi(p_s^{\alpha_s})\\
        &= p_1^{\alpha_1-1}(p_1-1)p_2^{\alpha_2-1}(p_2-1) ~\dots~ p_s^{\alpha_s-1}(p_s-1)
    \end{align*}
\end{defn}
\begin{thm}
    If $n$ is composite then there are integers $a$ and $b$ such that $n\mid ab$ but $n \nmid a$ or $n \nmid b$.
\end{thm}
\begin{proof}
    Since $n$ is composite then $n = x_1^{n_1} x_2^{n_2}~\dots y_1^{n_1^{'}}y_2^{n_2^{'}}~\dots $ where $ x,y$ are primes $\in (0,n)$ and $n_i's \geq 1~\forall i \in \mathbb{N}$. We have to prove the existence of the integers $a,b$ such that $n \mid ab$ but $n \nmid a$ or $n \nmid b$. \\
    We can constuct such integers given the prime factorization of $n$. If we define $a = x_1^{n_1} x_2^{n_2}~\dots $ and $ b = y_1^{n_1^{'}}y_2^{n_2^{'}}~\dots$ then we have satisfied the needed conditions.
\end{proof}
\begin{thm}
    If $p$ is a prime then $\sqrt{p}$ is not an rational number.
\end{thm}
\begin{proof}
    Suppose for contradiction, $\sqrt{p}$ is a rational number. Then there exist $x, y$ such that $\sqrt{p} = \frac{x}{y}$ and $(x, y) = 1$. Then $p = (\frac{x}{y})^2$ but $p$ is a prime hence the only factorization it has is $ p = p \times 1$ and factorization is unique by fundamental theorem of arithmetic. Hence contradiction, $\sqrt{p}$ is an irrational number. 
\end{proof}
\paragraph*{Ques.}If $p$ is a prime then prove that there do not exist nonzero integers $a$ and $b$ such that $a^2 = pb^2$.
\paragraph*{Ans.}If $a^2 = pb^2$ then $a = \pm \sqrt{p}b$ and using theorem 3, we can say $\sqrt{p}$ is an irrational number and the product of an irrational and an integer can never be an integer. Hence, there does not exist nonzero $a,b \in \mathbb{Z}$ such that $a^2 = pb^2$.
\hrule
\section*{Lecture 2}
\paragraph*{\textbf{$\mathbb{Z}/n \mathbb{Z}$} : Integers modulo n} Let $n$ be a fixed postive integer. Define a relation $R$ on $\mathbb{Z}$ as 
$$ aRb ~\text{{\em iff}}~ n \mid (b-a)$$ $R$ is the equivalence relation as can be verified. We call $a \equiv b \pmod n$, (read as: a is congruent to $b$ mod $n$) if $aRb$. \\
The equivalence class of $a$ is denoted by $\bar{a}$ and called {\em congruent class or residue class of $a \mod n$}.
\begin{align*}
    n\mid (b-a) &\implies b-a = nk~\text{for some}~k \in \mathbb{Z}\\
    &\implies b = a + kn ~\text{and}~ b \in \bar{a}
\end{align*}
For example: $\bar{0}$ = perfectly divisible by $n$. These residue classes partitions the $\mathbb{Z}$.\\
The set of all these equivalence classes under this equivalence relation will be denoted by $\mathbb{Z}/n\mathbb{Z}$, called {\em integers modulo n} or {\em integer mod n}.\\
The process of finding the equivalence class$\mod n$ of some integer $a$ is refered to as {\em reducing a$\mod n$}.

Addition and multiplication for elements of $\mathbb{Z}/n\mathbb{Z}$:
$$ \overline{a} + \overline{b} = \overline{a+b}~~~\text{and}~~~ \overline{a} . \overline{b} = \overline{ab}$$
This means that we can take any {\em representative} element from class $\bar{a}$ and any {\em representative} element from class $\bar{b}$ and then do usual addition (or multiplication) then find the class in which the result lies.

\textbf{For example}: If we take $\mathbb{Z}/2\mathbb{Z}$, then we have two classes $\bar{0}, \bar{1}$ (it's $0$ to $n-1$, $n=2$ here) then we can take $4$ and $7$ from $\bar{0}$ and $\bar{1}$ respectively. $4+7 = 11$ and $11$ lies in $\bar{1}$ class hence $\bar{0} + \bar{1} = \overline{4+7} = \bar{1}$.

The result is well defined and does not depend upon the choice of representatives as shown by the theorem below.
\begin{thm}
    
\end{thm}
\end{document}

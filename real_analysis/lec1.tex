\documentclass[12pt]{report}
\usepackage{fancyhdr}
\usepackage{amsmath}
\usepackage{amssymb}
\usepackage{amsfonts}
\usepackage{amsthm}
\usepackage[a4paper, width=150mm,top=25mm,bottom=25mm]{geometry}
\pagestyle{fancy}
\setlength{\headheight}{14.49998pt}
\fancyhead[L]{\small\leftmark}
\fancyhead[R]{\small\rightmark}
% \theoremstyle{definition}
\newtheorem{thm}{Theorem}
\newtheorem{lem}{Lemma}
\newtheorem{defn}{Definition}
\newtheorem*{rem}{Remark}
\newtheorem{cor}{Corollary}
\newtheorem{prop}{Proposition}


\begin{document}
\section*{Lecture 1}
\textbf{Remarks:} Basic ordering properties are assumed to be true.
\begin{defn}
    Boundedness: The subset $A \in R$ is said to be bounded above if $\exists M$ such that $M > x~~ \forall x \in A$. And it is bounded below if $\exists m$ such that $m < x~~ \forall x \in A$. If $A$ has both then it is called bounded.   
\end{defn}
\begin{defn}
    Least Upper Bound (lub) Axiom: If $A$ is nonempty subset of $R$ and it is bounded above, then $A$ has a least upper bound in $R$.
\end{defn}
\begin{thm}
    If $A$ is nonempty subset in $R$ and it is bounded below, then it has a greatest lower bound in $R$.
\end{thm}
\begin{proof}
    We first create a set $T$ of lower bounds of $A$ $$ T = \{m \mid m < x~~\forall x \in A\}$$ $T$ is non-empty since $A$ is bounded below. Now, we need to prove that there exits a supremum of $T$ which is also a lower bound of $A$.\\
    Since, set $T$ is bounded above by all the elements of set $A$, it should have a least upper bound, say $M$ such that $ M > m~~ \forall m \in T$. Also, every element of $A$ is an upper bound of $T$ hence by definition of supremum, we can say $M \leq x~~ \forall x \in A$ hence $M$ is the lower bound of $A$. This makes it the greatest lower bound.
\end{proof}
\begin{lem}
    Suppose $A \neq \phi$ and s = lub(A) then for any $y$ such that $y < s$, $\exists a \in A$ such that $ y < a \leq s$.
\end{lem}
\begin{proof}
    Suppose for contradiction, $\nexists$ any element $a$ such that $ y < a$. This means that $y \geq a, \forall a \in A$ $\implies$ $y$ is upper bound of set $A$. But $y$ is already less than least upper bound of set $A$. Hence contradiction. \\
    Therefore, $\exists$ $a \in A$ such that $ y < a \leq s$.   
\end{proof} 
\begin{thm}
    Archimedean Property: Given any positive real numbers $x,y~\exists~n \in N$ such that $ nx > y$.
\end{thm}
\begin{proof}
    Let a set $A = \{nx \mid n \in \mathbb{N}\}$. Suppose for contradiction $ nx \leq y$. Then $y$ is the upper bound of the set $A$. \\
    Let a $x > 0$, then $ y - x < y$ hence $y - x$ is not the upper bound of the set $A$. This means that $\exists m \in \mathbb{N}$ such that $y - x < mx$ $\implies$ $y < mx + x$ $\implies$ $y < (m+1)x$ which is impossible since $(m+1)x \in A$ and $y$ is upper bound of the A.\\
    $ nx > y$ is true.
\end{proof}
\begin{thm}
    If $A$ and $B$ are the two non empty bounded subsets of $R$, such that $x \leq y ~~ \forall x \in A$ and $\forall y \in B$ then $sup(A) \leq inf(B)$
\end{thm}
\begin{proof}
    Let $a$ be the supremum of $A$ and $b$ be the infimum of $B$. Therefore, \linebreak $a \geq x~~\forall x \in A$ and $ b \leq y~~ \forall y \in B$. Also, $A$ is bounded above by $B$ and elements of $B$ are the upper bound for $A$. Hence, $a \leq y ~~ \forall y \in B$. This means that $a$ is the lower bound of $B$ and $a$ is $sup(A)$. In other words, $sup(A) \leq inf(B)$. 
\end{proof}
\begin{thm}
    Given any two real number $a,b$ with $a<b$, $\exists$ $\mathbb{Q}$ between $a$ and $b$.
\end{thm}
\begin{proof}
    Since $b-a > 0$. Take two positive number $b-a$ and $1$ $\exists$ n $\in \mathbb{Z}$ such that $n(b-a) > 1$.\\ TBD
\end{proof}
\begin{thm}
    Any monotone increasing sequence of real numbers that is bounded above converges to some real number.
\end{thm}
\begin{proof}
    Let $x_n$ be a monotone increasing sequence in $\mathbb{R}$ that is bounded above hence there exits a $s$ such that $s = lub\{x_n \mid n \in \mathbb{N}\}$ \\
    Suppose $\epsilon > 0$ $\implies s - \epsilon < s$ and $s - \epsilon$ is not the upper bound of the $x_n$.\\ 
    Using lemma 1, we can say that $\exists x_\epsilon \in x_n$ such that $s - \epsilon < x_\epsilon < s$.\\
    Using monotone condition, for some $n_0 \in \mathbb{N}$, we have,
    $$ s - \epsilon < x_n < s < s + \epsilon~~~~\forall n > n_0 \in \mathbb{N}$$ Hence $ |x_n - s| < \epsilon$. $x_n$ converges to $s~~\forall n > n_0.$ 
\end{proof}
\begin{rem}
    Nested Interval theorem $\approx$ lub $\approx$ theorem $4$
\end{rem}
\begin{proof}
    TBD
\end{proof}
\begin{thm}
    Nested Interval theorem: Suppose $\{I_n\}$ is the sequence of closed and bounded non-empty intervals such that $I_1 \supset I_2 \supset I_3$ \dots then:
    \begin{enumerate}
        \item $\bigcap\limits_{n \geq 1} I_n \neq \phi$.
        \item If the sequence of the length of the intervals goes to $0$ then $\bigcap\limits_{n \geq 1} I_n = \{x\}$. 
    \end{enumerate}
\end{thm}

\begin{proof}
    Let $I_n$ be an interval $[a_n, b_n]$ with $a_m < b_n \forall m,n \in \mathbb{N}$ . Then $\forall n \in \mathbb{N}$, $a_n$ is the increasing sequence and $b_n$ is the decreasing sequence. $b_n$ is upper bound of $a_n$ hence, $a_n < inf(b_n)$.\\
    For $b_n$, $a_n$ is the lower bound of $b_n$ i.e. $sup(a_n)< b_n$. If we combine all inequalities, we get 
    $$ a_n \leq sup(a_n) \leq inf(b_n) \leq b_n ~~ \forall n \in \mathbb{N}$$
    Using density theorem, we can say that $\exists$ some $\mathbb{Q}$ between $sup(a_n)$ and $inf(b_n)$.\\
    Hence, $\bigcap\limits_{n \geq 1} I_n \neq \phi$.\\
    Let the length of the interval to be $L = |b_n - a_n|$. Suppose for contradiction, we have two elements in $\bigcap\limits_{n\geq 1}I_n$ instead of one, say $x$ and $y$. \\ The distance between $x$ and $y$ is $|y - x|$. Since, $L \to 0$ hence $\exists n \in \mathbb{N}$ such that for some $n_0 \geq n$, $|L| = |b_{n_0} - a_{n_0}|< \epsilon$ for some $\epsilon > 0$. Since $|L| \to 0$, we can choose $\epsilon$ such that it is smaller than $|y - x|$. Then, if interval contains any one of the point, it can not contain the other.
\end{proof}

\hrule
\section*{Lecture 2}
\begin{defn}
    Decimals representation of Real numbers: Let $z \in \mathbb{R}^+$ be given. Let $n_0$ be the largest integer such that $n_0 \leq z$. Let $n_1$ be the largest integer such that $n_0 + \frac{n_1}{10} \leq z$. As such, say $n_k$ is defined for some $k$. Let $n_{k+1}$ be the largest integer such that $n_0 + \frac{n_1}{10^1} + \frac{n_1}{10^2} + \dots + \frac{n_k}{10^k}+\frac{n_{k+1}}{10^{k+1}} \leq z$. Consider the set of all such finite sums, i.e. the set of all
    $$ z_k = n_0 + \frac{n_1}{10^1} + \frac{n_1}{10^2} + \dots + \frac{n_k}{10^k}+\frac{n_{k+1}}{10^{k+1}} \leq z$$
    This set has a supremum and that is $z$ itself. We symbolically write $z = n_0.n_1n_2 \dots $
\end{defn}
\begin{lem}
    Let $p$ be an integer $\geq 2$. If $0 \leq a_n \leq p-1$, where $a_n$ is an integer then $\sum\limits_{n=0}^{\infty} \frac{a_n}{p^n}$ converges to some $x$ in $[0,1]$.
\end{lem}
\begin{proof}
    Since $0 \leq a_n \leq p-1$, we can replace all $a_n$ with $p-1$ and then we have$$ \sum\limits_{n=1}^{\infty} \frac{a_n}{p^n} \leq (p-1)\sum\limits_{n=1}^{\infty} \frac{1}{p^n} = 1$$
    Therefore, the sequence is bounded, and it is monotonic increasing this means it converges to some $x \leq 1$ and $x$ is already positive. Hence it converges tp some $x$ in $[0,1]$. 
\end{proof}
\begin{lem}
    Conversely, given any  $ 0 \leq x \leq 1, \exists~ a_n \in \mathbb{Z}$ and $0 \leq  a_n \leq p-1$ such that $x = \sum\limits_{n=0}^{\infty} \frac{a_n}{p^n}$.
\end{lem}
\begin{proof}
    Suppose we have $0 < x \leq 1$ and $a_1$ is the largest integer such that $ \frac{a_1}{p} < x \leq 1$. Since $x$ is bounded above by $1$, we have $ a_1 < p$ $\implies$ $a_1 \leq p-1$ since $a_1$ is an integer. Similarly, find $a_2$ such that $\frac{a_1}{p} + \frac{a_2}{p^2} < x$.This can be achieved by Archimedean property. Also, note that $a_2 \leq p-1$, since we have 
    \begin{align*}
        \frac{a_1}{p} + \frac{a_2}{p^2} &< x < 1\\
        \frac{a_1}{p} + \frac{a_2}{p^2} \leq \frac{p-1}{p} + \frac{a_2}{p^2} &< 1~~~~ (a_1 \leq p-1)\\
        1 - \frac{1}{p} + \frac{a_2}{p^2} &< 1\\
        \frac{a_2}{p^2} &< \frac{1}{p}\\
        a_2 &< p \\
        a_2 &\leq p-1
    \end{align*}
Inductively, we can define $a_n$ as the largest integer with $a_n \leq p-1$ such that $\sum\limits_{i=1}^{n} \frac{a_i}{p_i} < x$. Since $a_n < p$
TBD
\end{proof}
Suppose $\{a_n\}$ is the bounded sequence in $\mathbb{R}$, we define two sets: 
\begin{align*}
    s_n &:= inf\{a_n, a_{n+1}, \dots \}\\
    S_n &:= sup\{a_n, a_{n+1}, \dots \}
\end{align*}
Notice that $ \underset{k}{inf}(\{a_n\}) \leq s_n \leq S_n \leq \underset{k}{sup}(\{s_n\})$
\begin{defn}
    Limit superior and limit inferior: Let $a_n$ be the bounded sequence of real numbers then
    \begin{align*}
        \lim inf(a_n) &= \lim_{n\to \infty} s_n\\
        \lim sup(a_n) &= \lim_{n\to \infty} S_n
    \end{align*}
\end{defn}
Limit superior is the supremum of all subsequential limits of $\{a_n\}$. Similarly, limit inferior is the infimum of all subsequential limits of $\{a_n\}$.

Note that $s_n$ is the increasing sequence and $S_n$ is the decreasing sequence and they are bounded on both sides. Hence, we can also say that using monotone convergence theorem
\begin{align*}
    \lim inf(a_n) &= \lim_{n\to \infty} s_n = sup(s_n)\\
    \lim sup(a_n) &= \lim_{n\to \infty} S_n = inf(S_n)
\end{align*}
\begin{thm}
    A sequence $\{a_n\}$ is bounded above iff limsup $a_n < \infty$ (is finite).
\end{thm}
\begin{proof}
    TBD
\end{proof}
\begin{thm}
    A sequence $\{a_n\}$ is bounded below iff liminf $a_n < \infty$ (is finite).
\end{thm}
\begin{proof}
    TBD
\end{proof}
\begin{thm}
    Given any sequence $\{a_n\}$ there exists a subsequence $\{a_{n_k}\}$ such that $a_{n_k} \to limsup(a_n)$.
\end{thm}
\begin{proof}
    TBD
\end{proof}
\begin{thm}
    Given any sequence $\{a_n\}$ there exists a subsequence $\{a_{n_k}\}$ such that $a_{n_k} \to liminf(a_n)$.
\end{thm}
\begin{proof}
    TBD
\end{proof}
\hrule
\section*{Lecture 3}
\begin{defn}
    Finite Set: A set is finite if there exists a bijection between the set and the $\{1,2, \dots, n\}$ for some $n \in \mathbb{N}$.
\end{defn}
\begin{thm}
    $\mathbb{N}$ is an infinite set.
\end{thm}
\begin{proof}
    Negation of above definition would be a set is infinite if there does not exists a bijection between $\{1,2, \dots, n\}$ and $\mathbb{N}$. Suppose a function $f(\{1,2, \dots, n\}) \to \mathbb{N}$. \\
    We can have a natural number $f(1)+f(2)+\dots+f(n) > f(i)~~\forall i \in \mathbb{N}$ that does not have a preimage in $\{1,2, \dots, n\}$ hence the map is not bijective.
\end{proof}
\begin{thm}
    A set is infinite iff there exists a one-one map from $\mathbb{N}$ to the set.
\end{thm}
\begin{proof}
    $\implies$) For $1 \in \mathbb{N}$ there exist a image in $X$ say $f(1)$. Now, take $2 \in \mathbb{N}$ such that there exist a image $f(2) \in X\backslash \{1\}$. This means that $f(1) \neq f(2)$. Since for every $n \in \mathbb{N}$ we can have $f(n)$ in $X\backslash\{1,2, \dots, n-1\}$ as $X$ is also infinite. Hence, we have constructed a one-one map from $\mathbb{N} \to X$.\\
    ($\impliedby$ Since $\mathbb{N}$ is infinite and we have a one-one mapping from $\mathbb{N} \to X$ therefore for each $n \in \mathbb{N}$, we have only one $f(n) \in X$ and every $n \in \mathbb{N}$ has a image (it's a map). Hence, $X$ is infinite.
\end{proof}
\begin{defn}
    Equivalent or Equipotent set: Two sets are equivalent or equipotent if there exists bijection between $X$ and $Y$.
\end{defn}
For example: Finite sets are equivalent to $\{1,2, \dots, n\}$ for some fixed $n \in \mathbb{N}$.
\begin{defn}
    Countably infinite set: A infinite set $X$ is said to be countably infinite if there exists a bijection between $X$ and $\mathbb{N}$. 
\end{defn}
\begin{defn}
    Uncountably infinite set: A set is said to be uncountably infinite if it is not countably infinite set. 
\end{defn}
\textbf{Example:} 
\begin{enumerate}
    \item Countably infinite: bijective map between $\mathbb{Z} \to \mathbb{N}.$ The map will look line
    $$ 
    n\to \begin{cases}
            \frac{n}{2}& n\in even\\
            \frac{-(n+1)}{2}&~~ n \in odd    
         \end{cases}
    $$
    \item $\mathbb{N} \times \mathbb{N} \to \mathbb{N}$ is also equivalent to $\mathbb{N}$ i.e. countably infinite. Map will be $ (m \times n) \to 2^m(2n-1)$ $m$ and $n$ are unique and $m$ such that $2^m$ is the maximum multiple of $2$.
\end{enumerate}
\begin{thm}
    If $A$ is an infinite subset of $\mathbb{N}$ then there exists bijection between $A$ and $\mathbb{N}$.
\end{thm}
\begin{proof}
    TBD
\end{proof}
\begin{cor}
    Monotone Subsequence theorem: Any sequence $\{x_n\}$ of real numbers has a monotone subsequence
\end{cor}
\begin{proof}
    We define "peak" as any element $x_m$ is called a peak if $x_m\geq x_n$ for all $n > m$. There cases can be two possible cases
    \begin{enumerate}
        \item Infinite peaks: This means that there exists $m_i's$ say $\{m_1, m_2, \dots\}$ such that $x_{m_i} > x_n$ for all $n> m_i ,~\text{and for all}~i \in \mathbb{N}$. We can arrange $m_i's$ in increasing order $m_1 < m_2 < \dots$ and $x_{m_1} > x_{m_2} > x_{m_3} > \dots$ is a decreasing subsequence.
        \item Finite peaks: ($0$ or some $n \in \mathbb{N}$). Assume the peaks are $ \{x_{m_1}, x_{m_2} , \dots, x_{m_n}\}$, this means that there exists some $s_1 = m_n+1, s \in \mathbb{N}$ such that $x_{s_1}$ is not a peak. Therefore, there also exists some $s_2 > s_1$ such that $x_{s_1} < x_{s_2}$. Because of finite peaks $x_{s_2}$ is also not a peak, hence for some $s_3 > s_2$, we have $x_{s_2} < x_{s_3}$. Proceeding with induction, we have $x_{s_1} < x_{s_2} < x_{s_3} < \dots$ is a increasing sequence. 
    \end{enumerate}
\end{proof}
\begin{thm}
    $\mathbb{Q}$ is a countably infinite set.
\end{thm}
\begin{proof}
    TBD
\end{proof}
\begin{thm}
    Any interval in $\mathbb{R}$ is an uncountable set.
\end{thm}
\begin{proof}
    TBD
\end{proof}
\begin{defn}
    Suppose $X \neq \phi$. A partial order on $X$ is a relation $R$ on $X$ such that $R$ is
    \begin{enumerate}
        \item Reflexive.
        \item Anti-symmetric $\implies$ $aRb, bRa \implies a=b$
        \item Transitive.
    \end{enumerate} 
\end{defn}
\textbf{Examples:}
\begin{enumerate}
    \item $R = "\leq"$ is a partial order.
\end{enumerate}
$\mathcal{P(A)}$ is power set of $A$ and $X,Y \subset A$ then $X \leq Y$ iff $X \subseteq Y$.
\hrule
\section*{Lecture 4}
\begin{defn}
    Given $E \subset X$ where $X$ is partially order set, we say $E$ is totally ordered if any two elements of $E$ are comparable i.e. if $e_1,e_2 \in E$, then $e_1 \leq e_2$ or $e_2 \leq e_1$. Totally ordered $\equiv$ linearly order $\equiv$ chain.
\end{defn}
\begin{defn}
    Upper bound of $E$: An element is $x \in X$ is called upper bound of $E$ if for any $x' \in E$, we have $x' \leq x$. $x$ is called the maximal element if $x' \geq x \implies x' = x$.\\
    For maximal element, $x$ should be an upper bound for set $E$ and $x$ should belong to $E$.
\end{defn}
Let $X \neq \phi$. $\mathcal{F}$ is a collection of subsets of $X$ (element of $\mathcal{P(\text{X})}$). An element $F \in \mathcal{F}$ is a upper bound for a subfamily $\mathcal{F'}$ of $\mathcal{F}$ provided every member of $\mathcal{F'}$ is a subset of $F$.\\
$F$ will be the maximal element of $\mathcal{F}$ if it is not a proper subset (means not contained in) of any member in $\mathcal{F}$. 
\begin{lem}
    Zorn's lemma: Let $X$ be a partially order set. If every totally ordered subset of $X$ is bounded above then $X$ has a maximal element.
\end{lem}
\begin{defn}
    Cardinality of $X:$ Two sets $A$ and $B$ have same cardinality if there exists a bijection between them. Set $X$ has cardinal number $\alpha$ means that there exists a set $Y$ equivalent to $X$ with number of elements equal to $\alpha$.   
\end{defn}
If $\alpha$ and $\beta$ are cardinal numbers of set $X$ and set $Y$ such that $\alpha \leq \beta$ then there exists a one-one mapping from $X \to Y$.
\begin{thm}
    Cantor-Schroeder-Bernstein theorem: If there exists a one-one mapping from $X \to Y$ and $Y \to X$ then there exists a bijection between $X$ and $Y$.
\end{thm}
\subsection*{Limits of functions}
\begin{defn}
    Limit point: A point $a \in \mathbb{R}$ is called a limit point of a set $X \subseteq \mathbb{R}$ if for every neighbourhood $(a - \epsilon, a+ \epsilon), \epsilon>0$ there exists $x \in X$ such that $a \neq x$.  
\end{defn}
For a function $f$ defined on $X \subseteq \mathbb{R}$, $f$ converges to some $l$ means that for every $\epsilon > 0$ there exists a $\delta > 0$ such that
$$ |f(x) - l| < \epsilon ~\forall x \in X \ni |x - a| < \delta$$ 
Limit of functions as limit of sequences-
\begin{prop}
    Let $f:X\to \mathbb{R}$ and let $a$ be a limit of $X$. Then $lim_{x \to a} f(x) = l$ if and only if for every sequence $\{x_n\}_{n \geq 1}$ in $X$ that converges to $a$ and $x_n \neq a$ for all $n$, the sequence $\{f(x_n)\}_{n \geq 1}$ converges to $l$. 
\end{prop}
A function $f$ is continuous on $X$ if it is continuous in every point in $X$.
\begin{defn}
    Continuity of $f$: A function $f$ is said to be continuous at some point $x \in X$ if for every $\epsilon > 0$ there exists a $\delta > 0$ such that for all $y \in X$ and $|y - x| < \delta$, we have $|f(y) - f(x)| < \epsilon$.
\end{defn}
A function $f$ is continuous at a limit point $a \in X$ if and only if $f(a)$ is defined and $ lim_{x \to a} f(x) = f(a)$.
\begin{prop}
    Let $f$ be a real valued funtion defined on subset $X$ of $\mathbb{R}$ and $a \in X$ is the limit point of $X$. Then $f$ is continuous at $a$ if and only if for every sequence $\{x_n\}_{n \geq 1}$ that converges to $a$ and $x_n \neq a$ for every $n$, we have $lim f(x_n) = f(lim x_n) = f(a)$. Continuous function preserve convergence (maps convergent sequence into convergent sequences).  
\end{prop}
\begin{thm}
    Bolzano intermediate value theorem: Let $I$ be an interval and $f: I \to \mathbb{R}$, if $a,b \in I$ and $\alpha \in \mathbb{R}$ satisfies $f(a) < \alpha < f(b)$ or $f(a) > \alpha > f(b)$ then there exists a point $c \in I$ between $a$ and $b$ such that $f(c) = \alpha$.
\end{thm}
\subsection*{Sequences of functions}
Let $X \subseteq \mathbb{R}$. If for every $n = 1,2, \dots$, we assigned a real valued function $f_n$ defined on $X$ then $\{f_n\}_{n \geq 1}$ is called sequence of  functions.\\
\textbf{Point-wise convergence}\\
A sequence is called point-wise convergent if for each $x \in X \subseteq \mathbb{R}$, the sequence $f_1(x), f_2(x), \dots$ of real numbers is convergent. \\
\textbf{Point-wise limit} \\
A function defined on $X$ is called a point-wise limit if for every $\epsilon > 0$ there exist $n_0 \in \mathbb{Z}$ depending on $x$ and $\epsilon$ such that for all $n \geq n_0$, we have $|f(x) - f_n(x)| < \epsilon$.

A series $\sum\limits_{n = 1}^{\infty} x_n$ of real numbers converges to $x \in \mathbb{R}$ if the sequence of partial sums $\{s_n\}_{n\geq 1}$ converges to $x$ where $s_n = \sum\limits_{k = 1}^n x_k$ ($n_{th}$ partial sum).\\
The limit $x = lims_n = lim_{n \to \infty}\sum\limits_{k = 1}^n x_k$ is called the sum of the series. In the case of functions, if for every $x \in X$, $\sum\limits_{n = 1}^{\infty} f(x_n)$ converges and if we define $f(x) = \sum\limits_{n = 1}^{\infty} f(x_n)$, the $f(x)$ is called the sum of the series $\sum\limits_{n = 1}^{\infty} f_n$.\\
\textbf{Uniform convergence}\\
A sequence of functions $\{f_n\}_{n \geq 1}$ defined on set $X \subseteq \mathbb{R}$ is said to be uniformly convergent on $X$ to $f$ if for every $\epsilon > 0$ there exists a $n_0 \in \mathbb{Z}$ (depending on $\epsilon$ only) such that for all $n \geq n_0$, we have $|f(x) - f_n(x)| < \epsilon$ for all $x \in X$.\\
Thus, every uniformly convergent sequence is pointwise convergent but converse is not true always.\\
\textbf{Uniform Convergence for series of functions}\\
The series of functions $\sum\limits_{n=1}^{\infty}$ converges uniformly on $X$ if the sequence $\{s_n\}_{n \geq 1}$ partial sum of functions, where $s_n(x) = \sum\limits_{k=1}^n f_k(x), x \in X$ converges uniformly on $X$.
\textbf{Cauchy Criterion of Uniform convergence}\\
The sequence of functions $\{f_n\}_{n\geq 1}$ defined on $X \in \mathbb{R}$ converges uniformly on $X$ if and only if given $\epsilon > 0$ there exists $n_0$ such that for all $x \in X$, and all $m \geq n_0,n\geq n_0$ we have $|f_m(x) -f_n(x)| < \epsilon$.
\begin{prop}
    Suppose $\{f_n\}_{n\geq 1}$ is sequence of continuous functions defined on $X$ and it converges uniformly to $f$ then $f$ is continous.
\end{prop}
\subsection*{Basic Topology in $\mathbb{R}$}
\begin{defn}
    Open sets in $\mathbb{R}$: A subset $G$ of $\mathbb{R}$ is said to be open if for every $x \in G$, there is a neighbourhood $(x -\epsilon, x+\epsilon), \epsilon > 0$ that is contained in $G$.
\end{defn}
\begin{defn}
    Open cover: A open cover of $X$ is a collection of $C = \{G_{\alpha} \mid \alpha \in I\}$ of open sets in $\mathbb{R}$ whose union contains the set $X$, $$ X \subseteq \bigcup\limits_{\alpha} G_{\alpha} $$ where $I$ is some indexing set.
\end{defn}
\begin{defn}
    Sub-cover: If $C'$ is a subcollection of $C$ such that the union of sets in $C'$ also contains the set $X$ then $C'$ is called subcover from $C$ of $X$. If the number of sets in $C'$ is finite then we call it a finite subcover. 
\end{defn}
\begin{defn}
    Compact set: A subset $X$ of $\mathbb{R}$ is said to be compact if every open cover of $X$ has a finite sub-cover. 
\end{defn}
\begin{prop}
    (Heine-Borel Theorem) Let $X$ be a set of real numbers. Then the following statements are equivalent:
    \begin{enumerate}
        \item $X$ is closed and bounded.
        \item $X$ is compact (every open cover has a finite subcover).
        \item Every infinite subset of $X$ has a limit point in $X$.
    \end{enumerate}
\end{prop}
\begin{proof}
    $1\implies 2$) Suppose $X = [a,b]$ a infinite interval which is closed and bounded. We define a set $S$
    $$ S = \{x \in [a,b] \mid \exists n \in I \ni [a,x] \subseteq \bigcup\limits_{i = 1}^{i = n} O_i\}$$

    $S$ is non-empty as we can always choose $n$ such that $S$ contains atleast $a$. Also, $b$ is the upper bound of the set $S$ and then it should possess the least upper bound property. Let $\lambda = sup(S)$.

    Basically, $S$ is a collection of all $x$ such that the interval $[a, x]$ has a finite subcover. Hence, we need to show that $\lambda$ is largest possible element of $S$ (i.e. $\lambda \in S$) for which $[a, \lambda]$ has a finite subcover and $\lambda = b$.

    Since, $\lambda = sup(S)$ and $\lambda < b$ hence, $\lambda \in X$. This means that there exists atleast one open set $O_{\beta}$ that contains $\lambda$ for  some $\beta \in I$. Hence, for some $\epsilon > 0$, the neighbourhood $(\lambda - \epsilon, \lambda + \epsilon) \in O_{\beta}$. Since, $\lambda - \epsilon$ is not the supremum of set $S$. Therefore, there exists a element $x \in S$ such that $\lambda > x > \lambda - \epsilon$ $\implies$ $x \in O_{\beta}$. Also, $x \in S$ hence, by definition of $S$ the finite subcovers of $[a, x]$ are $\{ O_1, O_2, \dots, O_{n}\}$ for some fixed $n \in I$. Hence, adding $O_{\beta}$ to this set $\{ O_{beta}, O_1, O_2, \dots, O_{n}\}$ is also finite and it is subcover of set $[a, \lambda]$ $\implies$ $\lambda \in S$.

    For proving $\lambda = b$, we have $\lambda \leq b$. Suppose for contradiction that $\lambda < b$ $\implies$ $b - \lambda > 0$. Define $y := \lambda + \frac{1}{2}\min(\epsilon, b-\lambda)$. This implies that belongs $y$ to neighbourhood of $\lambda$ $\implies$ $ y \in O_{\beta}$ and $y \in \bigcup\{ O_{\beta}, O_1, O_2, \dots, O_{n}\}$ $\implies y \leq \lambda$. But by definition $y > \lambda$. Hence, contradiction. Therefore, $y = b$ is the only possibility. This means that $X:= [a, b]$ has finite subcover hence $X$ is compact.
\end{proof}
\begin{proof}
    $2 \implies 3$) Let a infinite subset $G$ of $X$ where $X$ is compact set. Suppose for contradiction $G$ does not have any limit point in $X$ TBD.
\end{proof}
\begin{prop}
    Let $f$ be real-valued continuous function defined on closed and bounded interval $I = [a, b]$. Then $f$ is bounded on $I$ and it assumes its maximum and minimum values in the interval $I$ i.e. there are points $x_1, x_2 \in I$ such that $f(X_1) \leq f(x)  \leq f(x_2)$ for all $x \in X$.
\end{prop}

\paragraph*{Uniformly Continuous function}
Let $f$ be real-valued continuous function defined on $X$. Then $f$ is said to be uniformly continuous if given $\epsilon > 0$, there is a $\delta > 0$ such that for all $x,y \in X$ with $|x - y| < \delta$, we have $|f(x) - f(y)| < \epsilon$.
\begin{prop}
    If a real-valued function $f$ is continous on a closed and bounded interval $I$ then $f$ is uniformly continuous on $I$.
\end{prop}
\subsection*{Sequences in $\mathbb{R}$}
\begin{enumerate}
    \item A convergent sequence in $\mathbb{Q}$ is a Cauchy sequence in $\mathbb{Q}$.
    \item A Cauchy sequence in $\mathbb{Q}$ is bounded; in particular, every convergent sequence in $\mathbb{Q}$ is bounded.
    \item Limit of a sequence say $\{x_n\}_{n \geq 1}$ is unique.
\end{enumerate}
\textbf{Let $F_Q$ denote the set of all cauchy sequence in $\mathbb{Q}$.}
\begin{defn}
    A sequence $\{x_n\}_{n \geq 1}$ in $F_Q$ is said to be equivalent to a sequence $\{y_n\}_{n \geq 1}$ in $F_Q$ if and only if $\lim_{n \to \infty}|x_n - y_n| = 0$. Notation of equivalence is $\{x_n\}_{n \geq 1} \sim \{y_n\}_{n \geq 1}$.
\end{defn}
\begin{prop}
    If $\{x_n\}_{n\geq 1} \in F_Q$ then $\lim_{n \to \infty} x_n = x$ if and only if $\{x_n\} \sim \{x\}$, where $\{x\}$ denotes the constant sequence with each term is equal to $x$. 
\end{prop}
\begin{proof}
    Follows from definition of equivalence relation.\\
    Since, $\lim_{n \to \infty} |x_n - x| = 0$ (from definition) $\implies$ $\lim_{n \to \infty} x_n =  \lim_{n \to \infty} x = x$. Converse, is similar. 
\end{proof}
\begin{prop}
    If $\{x_n\}$ and $\{y_n\}$ are in $F_Q$ then so the sequencs $\{x_n + y_n\}$ and $\{x_ny_n\}$.
\end{prop}
\begin{proof}
    Apply cauchy criterion for each sequence and choose $\epsilon$ to be $\frac{\epsilon}{2}$. Then try to find $\epsilon$ bound for sequences $\{x_n + y_n\}$ and $\{x_ny_n\}$.
\end{proof}
\begin{prop}
    If $\{x_n\}$, $\{y_n\}$, $\{x'_n\}$ and $\{y'_n\}$ are in $F_Q$ and $\{x_n\}\sim \{x'_n\}$, $\{y_n\}\sim \{y'_n\}$ then $\{x_n + x'_n\}\sim \{y_n + y'_n\}$ and $\{x_ny_n\} \sim \{x'_ny'_n\}$.
\end{prop}
\begin{proof}
    For $\{x_n\}\sim \{x'_n\}$, $\{y_n\}\sim \{y'_n\}$ follows from writing modulus inequality $|a| - |b| \leq |a-b| \leq |a+b| \leq |a| + |b|$ and then applying sandwich theorem.

    For $\{x_ny_n\} \sim \{x'_ny'_n\}$, we know that cauchy sequence in $\mathbb{Q}$ are bounded. Hence, there exists a rational $K_1,K_2$ such that $|x_n| \leq K_1$ and $|y'_n| \leq K_2$ for all $n$. We can write 
    $$ |x_n - x'_n| < \frac{\epsilon}{2K_1}~\text{and}~|y_n - y'_n| < \frac{\epsilon}{2K_2}$$
    \begin{align*}
        |x_ny_n - x'_ny'_n| &= |x_ny_n - x_ny'_n + x_ny'_n - x'_ny'_n|\\
        &= |x_n(y_n - y'_n) + y'_n(x_n - x'_n)|\\
        &\leq |x_n||(y_n - y'_n) | + |y'_n||(x_n - x'_n)|\\
        &< K_1\left(\frac{\epsilon}{2K_1}\right) + K_2\left(\frac{\epsilon}{2K_2}\right)\\
        &< \frac{\epsilon}{2} + \frac{\epsilon}{2} < \epsilon
    \end{align*}
\end{proof}
\subsection*{Inequalities}
\begin{prop} \label{1}
    The function $f(x) = \frac{x}{1+x}, x \geq 0,$ is monotonically increasing. 
\end{prop}
\begin{proof}
    For some $x, y \geq 0$ and $x > y$, we have $\frac{1}{1+x} < \frac{1}{1+y}$ and $1-\frac{1}{1+x} > 1-\frac{1}{1+y}$ $\implies$ $ \frac{x}{1+x} > \frac{y}{1+y}$. 
\end{proof}
\begin{thm}
    For any two real numbers $x$ and $y$, this inequality holds
    $$ \frac{|x+y|}{1+|x+y|} \leq \frac{|x|}{1+|x|} + \frac{|y|}{1+|y|}$$
\end{thm}
\begin{proof}
    Using previous proposition, we can say if $|x+y| \leq |x| + |y|$ and the sequence $\frac{x}{1+x}, x \geq 0$ is monotonically increasing then
    $$ \frac{|x+y|}{1+|x+y|} \leq \frac{|x| + |y|}{1+ |x| + |y|} = \frac{|x|}{1+|x| + |y|} + \frac{|y|}{1+|x| + |y|} \leq \frac{|x|}{1+|x|} + \frac{|y|}{1+|y|}$$ 
\end{proof}
\begin{prop}
    Generalised AM-GM inequality: If $a > 0$ and $b > 0$ and if $0 \leq \lambda \leq 1$ is fixed, then
    $$ a^{\lambda}b^{1-\lambda} \leq \lambda a + (1-\lambda)b$$ 
\end{prop}
\begin{proof}
    Since, $ y = ln(x)$ is concave, then
    \begin{align*}
        ln(\lambda a + (1-\lambda)b) &\geq \lambda ln(a) + (1-\lambda)ln(b)\\
        ln(\lambda a + (1-\lambda)b) &\geq ln(a^{\lambda} b^{1-\lambda})
    \end{align*}
Since $e^x$ is increasing function, we have
$$ a^{\lambda} b^{1-\lambda} \leq \lambda a + (1-\lambda)b$$
\end{proof}
\begin{rem}
    When $x \geq 0, y \geq 0$ and $p > 1$ and $\frac{1}{p} + \frac{1}{q} = 1$, we have 
    $$ xy \leq \frac{1}{p}x^p + \frac{1}{q}y^q$$
\end{rem}
\begin{proof}
    Replace $ a = x^p$, $b = y^q$ and take $\lambda = \frac{1}{p} \implies 1-\lambda = 1- \frac{1}{p} = \frac{1}{q}$
\end{proof}
\begin{thm}
    (Holder's inequality) Let $x_i \geq 0$ and $y_i \geq 0$ for $i = 1,2, \dots, n,$ and suppose that $p > 1$ and $q > 1$ are such that $\frac{1}{p} + \frac{1}{q} = 1$. Then
    $$ \sum\limits_{i=1}^n x_iy_i \leq \left(\sum\limits_{i=1}^nx_i^p\right)^{1/p} \left(\sum\limits_{i=1}^ny_i^q\right)^{1/q}$$
    In the special case when $p = q = 2$, the above inequality reduces to 
    $$ \sum\limits_{i=1}^n x_iy_i \leq \left(\sum\limits_{i=1}^nx_i^2\right)^{1/2} \left(\sum\limits_{i=1}^ny_i^2\right)^{1/2}$$
    This is called \textbf{Cauchy-Schwarz inequality.}
\end{thm}
\begin{proof}
    For the case of $x_i = 0$ and $y_i = 0$, it is trivially true.\\
    For the case of $x_i > 0$ and $y_i > 0$, we can write the given inequality as 
    $$ \sum\limits_{i = 1}^{n} \left(\frac{x_i}{\left(\sum\limits_{i = 1}^{n} x_i^p\right)^{1/p}} \frac{y_i}{\left(\sum\limits_{i = 1}^{n} y_i^q\right)^{1/q}} \right)\leq 1 $$
    Replace $ x_i' = \frac{x_i}{\left(\sum\limits_{i = 1}^{n} x_i^p\right)^2}$ and $y_i' = \frac{y_i}{\left(\sum\limits_{i = 1}^{n} y_i^q\right)^2} $, also $x_i' \geq 0$ and $y_i' \geq 0$, we get $ \sum\limits_{n = 1}^{n} x_i'y_i' \leq 1$. \\
    Now,  apply the Youngs inequality for $i = 1, 2, \dots, n$ and sum them up to get
    $$ \sum\limits_{i = 1}^{n} x_i'y_i' \leq \frac{\sum x^p}{p} + \frac{\sum y^q}{q}$$
    Since, $x_i > 0$ and $y_i > 0$ hence $\sum\limits_{n=1}^{n}x'^p \neq 0 \neq \sum\limits_{n=1}^{n}y'^q $ and it is equivalent to prove it for (some constant)
    $$ \sum\limits_{n=1}^{n}x'^p = 1 = \sum\limits_{n=1}^{n}y'^q $$
    Hence, we have
    $$ \sum\limits_{i = 1}^{n} x_i'y_i' \leq \frac{1}{p} + \frac{1}{q} = 1$$
    This proves the required inequality.
\end{proof}
\begin{thm}
    (Minkowski's inequality) Let $x_i \geq 0$ and $y_i \geq 0$ for $ i = 1,2,\dots, n$ and suppose that $p \geq 1$. Then
    $$ \left(\sum\limits_{i =1}^n(x_i + y_i)^p\right)^{1/p} \leq \left(\sum\limits_{i =1}^nx_i^p\right)^{1/p} + \left(\sum\limits_{i =1}^ny_i^q\right)^{1/q}$$  
\end{thm}
\begin{proof}
    If $p = 1$, it is trivially true. So, assume $p > 1$, we have
    $$ \sum\limits_{i=1}^n(x_i + y_i)^p = \sum\limits_{i=1}^{n} x_i(x_i + y_i)^{p-1} + \sum\limits_{i = 1}^n y_i(x_i + y_i)^{p-1}$$
    Apply Holder's inequality for both terms on RHS
    \begin{align*}
        \sum\limits_{i=1}^n(x_i + y_i)^p &\leq \left(\sum\limits_{i=1}^n x_i^p \right)^{1/p} \left(\sum\limits_{i=1}^n (x_i + y_i)^{(p-1)q}\right)^{1/q} \\
        &+ \left(\sum\limits_{i=1}^n y_i^p \right)^{1/p} \left(\sum\limits_{i=1}^n (x_i + y_i)^{(p-1)q}\right)^{1/q}\\
        &\leq \left[\left(\sum\limits_{i=1}^n x_i^p\right)^{1/p} + \left(\sum\limits_{i=1}^n y_i^p\right)^{1/p}\right] \left(\sum\limits_{i=1}^n(x_i + y_i)^p\right)^{1/q}
    \end{align*}
    Divide both sides by $\left(\sum\limits_{i=1}^n(x_i + y_i)^p\right)^{1/q}$ as it is $\neq 0$ and we will get the required inequality. For the case of $\left(\sum\limits_{i=1}^n(x_i + y_i)^p\right)^{1/q} = 0$, proof is self-evident. 
\end{proof}
\begin{thm}
    (Minkowski Inequality for infinite sums) Suppose that $p \geq 1$ and let $\{x_n\}_{n \geq 1}$ and $\{y_n\}_{n \geq 1}$ be sequences of non-negative terms such that $\sum\limits_{n=1}^{\infty} x_n^p$ and $\sum\limits_{n=1}^{\infty} y_n^p$ are convergent. Then $\sum\limits_{n=1}^{\infty} (x_n + y_n)^p$ is convergent. Moreover,
    $$ \left(\sum\limits_{n =1}^{\infty}(x_i + y_i)^p\right)^{1/p} \leq \left(\sum\limits_{n =1}^{\infty}x_i^p\right)^{1/p} + \left(\sum\limits_{n =1}^{\infty}y_i^q\right)^{1/q} $$
\end{thm}
\begin{proof}
    For any positive integer $m$, we can use Minkowski inequality for finite sums 
    \begin{align*}
        \left(\sum\limits_{n =1}^m(x_i + y_i)^p\right)^{1/p} &\leq \left(\sum\limits_{n =1}^mx_i^p\right)^{1/p} + \left(\sum\limits_{n =1}^my_i^q\right)^{1/q}\\
        &\leq \left(\sum\limits_{n =1}^{\infty}x_i^p\right)^{1/p} + \left(\sum\limits_{n =1}^{\infty}y_i^q\right)^{1/q}
    \end{align*}
    Since, the sequence $\{(\sum\limits_{n =1}^m(x_i + y_i)^p)^{1/p}\}$ is increasing sequence(w.r.t m) of nonnegative real numbers and is bounded above by the sum in above equation. Hence, it is convergent and as $m \to \infty$ the limit is also bounded above by the sum. This proves the inequality.
\end{proof}
\begin{thm} \label{22}
    Let $p > 1$. For $ a \geq 0$ and $ b \geq 0$, we have
    $$ (a + b)^p \leq 2^{p-1}(a^p + b^p)$$
\end{thm}
\begin{proof}
    TBD
\end{proof}
\subsection*{Metric Spaces}
\begin{defn}
    Let $X \neq \phi$. A metric on $X$ is a function $d:X \times X \to [0, \infty)$ such that 
    \begin{enumerate}
        \item d(x,y) = 0 $\implies x = y$
        \item d(x,y) = d(y,x)
        \item d(x,y) $\leq$ d(x,z) + d(z,y) for any $x,y,z \in X$.
    \end{enumerate}
\end{defn}
The metric space in which every cauchy sequence converges is called "complete" metric space.
\subsection*{Examples of metric spaces}
\begin{enumerate}
    \item \textbf{The space of all bounded sequences.} Let $X$ be the set of all infinite sequences of numbers that are bounded $\{x_i\}_{i \geq 1}$ such that $sup|\{x_i\}| < \infty$. The metric on $X$ is defined as $d(x,y)= sup_i|x_i - y_i| ~(\leq sup_i|x_i - z_i| + sup_i|y_i - z_i|).$
    \item \textbf{The space $l_p$.} Let $X$ be the set of all sequences $x = \{x\}_{i \geq 1}$ such that 
    $$ \left(\sum\limits_{i = 1}^{\infty} |x_p|^p\right)^{1/p} < \infty,~~~~p \geq 1$$
    If $\{x\}_{n \geq 1}$ and $\{y\}_{n \geq 1}$ are two sequences belong to $X$ then we define the metric as
    $$ d(x,y) = \left(\sum\limits_{i = 1}^{\infty} |x_i - y_i|^p\right)^{1/p}$$
    Minkowski inequality for infinite sequences makes it a suitable metric.
    \item \textbf{The space of bounded functions.} Let $S$ be any nonempty set and $\mathcal{B(S)}$ denote the set of all real or complex-valued functions on $S$, each of which is bounded. i.e. 
    $$ \sup\limits_{x \in S} |f(x)| < \infty$$
    It can also be shown that $\sup\limits_{x \in S} |f(x) - g(x)| < \infty$. We defined metric as $d(f,g) = \sup\limits_{x \in S}|f(x) - g(x)|,~~~~ f,g \in \mathcal{B(S)}$. The metric $d$ is called \textbf{uniform metric} or \textbf{supremum metric}.
    \item \textbf{The space of continuous functions.} Let $X$ be the set of all continuous functions defined on $[a,b]$, an interval in $\mathbb{R}$. For $f,g \in X$, define
    $$ d(f,g) = \sup\limits_{x\in [a,b]} |f(x) - g(x)|$$
    We can also defined another metric on the set $X$ as
    $$ d(f,g) = \int_a^b |f(x) - g(x)| dx$$
    Assume $h(x) \in X = C[a,b]$, we have
    \begin{align*}
        d(f,g) &= \int_a^b |f(x) - h(x) + h(x) - g(x)| dx\\
        &\leq \int_a^b \left(|f(x) - h(x)| + |h(x) - g(x)| \right)dx\\
        &\leq d(f,h) + d(h,g)
    \end{align*}
    \item \textbf{Metric on extended real line.} Let $X = \mathbb{R} \cup \{\infty\} \cup \{-\infty\}$. Define $f: X \to \mathbb{R}$ by the rule
    $$ f(x) = \begin{cases}
        \frac{x}{1+|x|} & \text{if } -\infty < x < \infty \\
        1 & \text{if } x = \infty\\
        -1 & \text{if } x = -\infty
    \end{cases}
    $$
    We define metric using $f$ as 
    $$ d(x,y) = |f(x) - f(y)|, ~~ x,y \in X$$
    \begin{proof}
        For injectivity, if $f(x) = f(y)$, we need to prove that $x = y$. Notice that $xy > 0$ is the only possible case. Since, $xy < 0$ ($x$ and $y$ are of opposite sign) is not possible because then $f(x) \neq f(y)$.
        \begin{align*}
            \frac{x}{1 + |x|} &= \frac{y}{1+|y|}\\
            x + x|y| &= y + y|x|\\
            x &= y~~~~\text{$\because$ xy $>$ 0}
        \end{align*}
        Hence, $f$ is one - one function. The metric $d$ is positive, $d(x,y) = d(y,x)$ $d = 0 \implies f(x) = f(y) \implies x = y$ ($\because f$ is one-one). On the other hand, if $x = y$ then it is obvious that $f(x) = f(y) \implies d = 0$. \\
        For triangle inequality, we can write
        \begin{align*}
            \left|\frac{x}{1+|x|} - \frac{1}{1+|y|}\right| &= \left|\frac{x}{1+|x|} - \frac{z}{1+|z|} + \frac{z}{1+|z|} - \frac{y}{1+|y|}\right| \\
            &= \left| \left(\frac{x}{1+|x|} - \frac{z}{1+|z|}\right) + \left(\frac{z}{1+|z|} - \frac{y}{1+|y|}\right)\right|\\
            &\leq \left|\frac{x}{1+|x|} - \frac{z}{1+|z|}\right| + \left|\frac{z}{1+|z|} - \frac{y}{1+|y|}\right|\\
            &\leq d(x,z) + d(z,y)
        \end{align*}
        Hence, it is a metric.
    \end{proof}
    \item In the above example, we can choose $f(x) = \tan^{-1} x$. Since, it is one - one function, and $\tan^{-1} (\infty) = \pi/2, \tan^{-1} (-\infty) = -\pi/2$. Then the metric will be
    $$ d(x, y) = |\tan^{-1} x - \tan^{-1} y|,~~~x,y \in X$$
    \item Let $d$ be a metric on nonempty set $X$. The function defined as 
    $$ e(x,y) = \min\{1, d(x,y)\},~~~~\forall x,y \in X$$ 
    is a metric.
    \begin{proof}
        The above expression says two things $e(x,y) \leq 1$ and $e(x,y) \leq d(x,y)$. We need to prove that $e(x,y) \leq e(x,z) + e(z,y)$. Since, if either of the terms on RHS or both are less than $1$, the inequality is satisfied because $e(x,y) \leq 1$. If both the terms on RHS are less than $1$ then, we have 
        $$ e(x,y) \leq \min\{1, d(x,y)\} \leq d(x,y) \leq d(x, z) + d(z,y) = e(x,z) + e(z,y)$$
    \end{proof}
\end{enumerate}
\begin{prop}
    Let $(X,d)$ be a metric space. Define $d':X \times X \to \mathbb{R}$ by 
    $$ d'(x,y) = \frac{d(x,y)}{1 + d(x,y)}$$
    Then $d'$ is a metric on $X$. Besides, $d'(x,y) < 1$ for all $x,y \in X$.
\end{prop}
\begin{proof}
    All axioms of metric spaces are satisfied, we just need to show the triangle inequality. We need to prove that $d'(x,y) \leq d'(x,z) + d'(z,y)$. Since, we know that $\frac{x}{1+x}, x\geq 0$ is monotonically increasing sequence \ref{1}. And we also know that $d(x,y) \leq d(x,z) + d(z,y)$. Hence, combining these two arguments, we have
    \begin{align*}
        d'(x,y) =  \frac{d(x,y)}{1 + d(x,y)} &\leq \frac{d(x,z) + d(z,y)}{1 + d(x,z) + d(z,y)}\\
        &= \frac{d(x,z)}{1 +  d(x,z) + d(z,y)} + \frac{d(z,y)}{1 +  d(x,z) + d(z,y)}\\
        &\leq \frac{d(x,z)}{1 +  d(x,z)} + \frac{d(z,y)}{1 +  d(z,y)}(\because d(x,z),d(z,y) \geq 0)\\
        &= d'(x,z) + d'(z,y)
    \end{align*}
\end{proof}
\paragraph*{Examples}
\textbf{The space of all sequences of numbers. } Let $X$ be the space of all sequence of numbers. Let $x = \{x_i\}_{i \geq 1}$ and $y = \{y_i\}_{i \geq 1}$ be elements of $X$. Define
$$ d(x,y) = \sum\limits_{i=1}^{\infty}\frac{1}{2^i}\frac{|x_i - y_i|}{1 + |x_i - y_i|}$$
is a metric.
\begin{proof}
    All the metric space axioms are satisfied, we only need to show for triangle inequality. Since, we know that $\frac{1}{2}, \frac{1}{2^2}, \dots$ are positive numbers hence we can conclude that
    $$ \frac{1}{2}\frac{|x|}{(1 + |x|)} < \frac{1}{2}\frac{|y|}{(1 + |y|)},~~~~x < y$$
    We can sum both the sides to a similar inequality as above and the fact that $|x_i - y_i| \leq |x_i - z_i| + |z_i - y_i|$ makes it clear to get the required result.    
\end{proof}
\begin{rem}
    The above expression is not true in $\mathbb{R}^2$. In $\mathbb{R}^2$, if we define metric $d(x,y) = |x_2 - y_2|$ and  take $x = (0,0)$ and $y = (1,0)$ then we have $d(x,y) = 0$ but $x \neq y$. Hence, $d$ is not a metric in $\mathbb{R}^2$.
\end{rem}
\begin{defn}
    Pseudometric: Let $X$ be a nonempty set. A pseudometric on $X$ is a mapping of $X \times X \to \mathbb{R}$ that satisfies the axioms
    \begin{enumerate}
        \item $d(x,y) \geq 0$.
        \item $d(x,y) = 0~~\text{if}~~ x = y$
        \item $ d(x,y) = d(y,x),~~~\forall x,y \in X$
        \item $d(x,y) \leq d(x,z) + d(z,y),~~~\forall x,y,z \in X$.
    \end{enumerate}
\end{defn}
\paragraph*{Example}
Let $X$ be the set of all Riemann integrable functions on $[a,b]$. For $f,g \in X$, define
$$ d(f,g) = \int_a^b |f(x) - g(x)| dx$$
This is a pseudometric on $X$ but it is not a metric on $X$.\\
\textbf{Counter example}\\
Let
$$ 
f(x) = \begin{cases}
        0 & \text{if}~~x=a\\
        1 & \text{if}~~a < x \leq b
    \end{cases}
$$
and $g(x) = 1~~\forall x \in [a,b]$. We can see that $d(f,g) = 0$ but $f \neq g$.
\paragraph*{A equivalence relation on a pseudometric space generates a metric space. } Let $(X,d)$ be a pseudometric space. Define a relation $R$ on $X$ by 
$$ xRy~~\text{if and only if}~~ d(x,y) = 0$$
Then the relation $R$ is an equivalence relation on $X$ (for transitivity, $xRy$,$yRz$ then use triangle inequality for $d$ to get $d(z,x) \leq 0 \implies d(z,x) = 0$).\\
The set of all equivalence classes denoted by $X/R$ forms a metric space with the metric defined as
$$ \tilde{d}([x],[y]) = d(x,y)$$
where $x \in [x],y \in [y]$ and the metric is independent of the choice of the representatives.
\begin{proof}
    (For the statement that the metric is well defined and it does not depend on the choice of the representative.)\\
    Let $x,x' \in [x]$ and $y,y' \in [y]$. We need to show that $d(x,y) = d(x',y')$. Using triangle inequality, we can say
    \begin{align*}
        d(x,y) &\leq d(x,x') + d(x',y') + d(y',y)\\
        d(x,y) &\leq d(x',y')~~(\because x \sim x', y \sim y')
    \end{align*}
    By similar argument, we can show $d(x',y') \leq d(x,y)$. Hence, $d(x',y') = d(x,y)$ and the metric does not depend upon the choice of representative. It is easy to show that this is a metric.
\end{proof}
\subsection*{Sequences in metric spaces}
\begin{defn}
    Let $d$ be a metric on a set $X$ and $\{x_n\}$ be a sequence in the set $X$. An element $x \in X$ is said to be a limit of $\{x_n\}$ if, for every $\epsilon > 0$, there exist a natural number $n_0$ such that for all $n \geq n_0$,
    $$ d(x_n, x) < \epsilon$$
    Hence, $\{x_n\}$ converges to $x$.
\end{defn}
\begin{rem}
    The limit $x$ of the sequence is always unique.
\end{rem}
\begin{proof}
    Suppose there exists two limit points for the sequence $\{x_n\}$, $x_1$ and $x_2$. This means for given $\epsilon > 0$, there exists $n_1,n_2$ such that $d(x_n , x_1) < \frac{\epsilon}{2}~~\forall n_1 \geq n_0$ and $d(x_n , x_2) < \frac{\epsilon}{2}~~\forall n_2 \geq n_0$. Using triangle inequality, $d(x_1, x_2) \leq d(x_2,x_n) + d(x_n, x_1) = \epsilon$. Hence $x_1 = x_2$.
\end{proof}
\paragraph*{Examples}
\begin{enumerate}
    \item Let $X \in \mathbb{R}^n$ with metric
    $$ d(x,y) = d_p(x,y) = \left(\sum_{j=1}^n|x_j - y_j|^p\right)^{1/p},$$
    where $x = (x_1, x_2, \dots,x_n)$ and $y = (y_1,  y_2, \dots,y_n)$ are in $\mathbb{R}^n$ and $p \geq 1$. Let $x^{(k)} = (x_1^{(k)},x_2^{(k)},\dots,x_n^{(k)})$ be a sequence in $\mathbb{R}^n$ converges to $x = (x_1,x_2,\dots,x_n)$. 
    $$ \lim_{k\to \infty}\left(\sum\limits_{j=1}^{n}|x_j^{(k)} - x_j|\right)^{1/p} = \epsilon~~\forall k \geq k_0$$
    For every $\epsilon > 0$, there exists an integer $k_0(\epsilon)$ such that $|x_j^{(k)} - x_j| < \epsilon~~\forall k \geq k_0$. Hence, the sequence converges coordinatewise.\\
    On the other hand, if for each $j = 1, 2, \dots, n$ there exists the integers $k_j$ such that $|x_j^{(k)} - x_j| < \frac{\epsilon}{n^{1/p}}$ for all $k \geq k_j$ then $k' = \max(k_1, k_2,\dots, k_n)$ we have
    $$ \left(\sum\limits_{j=1}^{n}|x_j^{(k)} - x_j|\right)^{1/p} < \epsilon~~\forall k \geq k'$$
    \textbf{Thus, convergence of sequences in $(\mathbb{R}^n,d_p)$ is equivalent to coordinatewise convergence.}
    \item If metric is $ d_{\infty}(x_n,x) = \max\limits_j|x_j^{(k)} - x_j|$ ($\{x_n\}_{n \geq 1}$ same as above example) then for every $\epsilon > 0$ there exists an integer $k_0$ such that we have $\max\limits_j|x_j^{(k)} - x_j| < \epsilon$ for all $k \geq k_0$. Hence, $|x_j^{(k)} - x_j| < \epsilon$ for each $j = 1,2, \dots,n$. On the other hand, if for each $ j = 1,2,\dots, n$, there exists an integer $k_j$ such that $|x_j^{(k)} - x_j| < \epsilon$ for $ k \geq k_j$ then we can choose $ k' = \max(k_1, k_2, \dots, k_n)$ so that
    $$ \max\limits_j|x_j^{(k)} - x_j| < \epsilon~~~\forall k \geq k'$$
    Hence, in $(\mathbb{R}^n,d_{\infty})$ convergence \textbf{implies} coordinatewise convergence.
    \item Similarly, convegence in the metric space with discrete metric also \textbf{implies} coordinatewise convergence.
    \item For space of all sequences with metric 
    $$ d(x,y) = \sum\limits_{j = 1}^{\infty} \frac{1}{2^j}\frac{|x_j^{(k)} - x_j|}{1 + |x_j^{(k)} - x_j|}$$
    TBD
    \item In this example \textbf{coordinatewise convergence does not imply convergence.}\\
    Let $X = l_p(p \geq 1)$ and let $d(x,y) = (\sum_{k=1}^{\infty}|x_k - y_k|^p)^{1/p}$, where $\{x_k\}_{k \geq 1}$ and $\{y_k\}_{k \geq 1}$ are in $l_p$. Let $\{\{x_k\}_{k\geq 1}\}_{n \geq 1}$ be a sequence in $l_p$ that converges to $x \in l_p$
    $$ \lim_{n \to \infty}d(x^{(n)}, x) = \lim_{n \to \infty}\left(\sum_{k=1}^{\infty}|x_k^{(n)} - x_k|^p\right)^{1/p}$$
    Then for $\varepsilon > 0$ there exists a positive integer $n_0(\varepsilon)$ such that $$ \left(\sum_{k=1}^{\infty}|x_k^{(n)} - x_k|^p\right)^{1/p}$$
    for all $ n \geq n_0(\varepsilon)$. Since $\varepsilon > 0$, it follows that $ \lim_{n \to \infty}x_k^{(n)} = x_k$ for each $k$.\\
    The converse is not true. Counter example: Let $x_k^{(n)} = x_k + \mathcal{\delta}_{kn}$ where 
    $$ \mathcal{\delta}_{kn} = \begin{cases}
        0 & \text{if}~~k \neq n\\
        1 & \text{if}~~k = n\\
    \end{cases} 
    $$
    Here $ |x_k^{(n)} - x_k| = \mathcal{\delta}_{kn} = 0$ if $n > k$. Hence, $\lim_{n \to \infty} x_k^{(n)} = x_k$ for each $k$. But
    $$ d(x^{(n)},x) = \left(\sum_{k=1}^{\infty}|x_k^{(n)} - x_k|^p\right)^{1/p} = 1 $$
    for each $n$.. Thus $x^n \nrightarrow x$ in the $l_p$ metric.\\
    But the below theorem holds.
\end{enumerate}
\begin{thm}
    Let $\{x^{(n)}\}{n \geq 1}$ be a sequence in $l_p$ such that $\lim_{n \to \infty}x_k^{(n)} = x_k$ for each $k$, where $x = \{x_k\}_{k \geq 1}$ is an element of $l_p$. Suppose also that for every $ \varepsilon > 0$ there exists an integer $m_0(\varepsilon)$ such that
    $$ \left(\sum\limits_{k = m+1}^{\infty} |x_k^{(n)}|^p \right)^{1/p} < \varepsilon$$ for $m \geq m_0(\varepsilon)$ and for all $n$. Then $\lim_{n \to \infty}d(x^{(n)}, x) = 0$.
\end{thm}
\begin{proof}
    TBD(cumbersome to write)
\end{proof}
\subsection*{Cauchy Sequences}
\begin{defn}
    Let $d$ be a metric on set $X$. A sequence $\{x_n\}_{n \geq 1}$ in the set $X$ is said to be cauchy if, for every $\varepsilon > 0$ there exists a positive integer $n_0$ such that $d(x_n,x_m)< \varepsilon$ for all $n \geq n_0$ and $m \geq n_0$.
\end{defn}
\begin{rem}
    In $\mathbb{R}$ and $\mathbb{C}$, a sequence is convergent if and only if it is cauchy.
\end{rem}
\begin{rem}
    In general metric spaces, every convergent sequence is cauchy, converse is not necessarily true.
\end{rem}
\begin{prop}
    A convergent sequence in a metric space is a Cauchy sequence.
\end{prop}
\begin{proof}
    Assume $X$ is a metric space and $d$ is the metric defined on $X$. Let $\{x_n\}_{n \geq 1}$ is a convergent sequence and $x$ is the limit. For given $\varepsilon > 0$ there exists a positive integer $n_0$ such that $d(x_n, x) < \varepsilon$ for all $n \geq n_0$. Suppose $n,m \geq n_0$ such that $d(x_n, x) < \varepsilon$ and $d(x_m, x) < \varepsilon$. Then using triangle inequality, we can say $d(x_n, x_m) < \varepsilon$ hence the sequence is cauchy. 
\end{proof}
Converse, is however, not true. Counter example: for the set $X$ with metric $|x - y|$, sequence of rationals such as $1.4, 1.41, 1.414, \dots $ limit is $\sqrt{2}$ (means its converging and therefore its cauchy) which is irrational hence sequence is converging to a limit which does not belong to the set of rationals.\\
Hence, Cauchy sequence does not need to converge to a point in the space.
\begin{defn}
    A metric space $(X,d)$ is said to be complete if every Cauchy sequences in $X$ is convergent.
\end{defn}
\textbf{Example:} $\mathbb{R},\mathbb{C}$ and $\mathbb{R}^n$ are complete with their usual metrics. A set of rationals with metric $|x - y|$ for all $x,y \in \mathbb{Q}$ is incomplete as there exits a sequence in $\mathbb{Q}$ ($1.4, 1.41, 1.414, \dots)$ converges to $\sqrt{2}$ which does not belong to $\mathbb{Q}$.
\begin{defn}
    Let $\{x_n\}_{n\geq 1}$ be a given sequence in a metric space $(X,d)$ and let $\{n_k\}_{k\geq 1}$ is sequence of positive integer $n_1 < n_2 < \dots$ Then the sequence $\{x_{n_k}\}_{k \geq 1}$ is called a subsequence of $\{x_n\}_{n\geq 1}$. If the subsequence converges then its limit is called subsequential limit. 
\end{defn}
\begin{rem}
    A sequence $\{x_n\}_{n\geq 1}$ in $X$ converges to $x$ if and only if all of its subsequences converges      $x$.
\end{rem}

\end{document}


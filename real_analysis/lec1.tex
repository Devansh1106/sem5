\documentclass[12pt]{report}
\usepackage{fancyhdr}
\usepackage{amsmath}
\usepackage{amssymb}
\usepackage{amsfonts}
\usepackage{amsthm}
\usepackage[a4paper, width=150mm,top=25mm,bottom=25mm]{geometry}
\pagestyle{fancy}
\setlength{\headheight}{14.49998pt}
\fancyhead[L]{\small\leftmark}
\fancyhead[R]{\small\rightmark}
% \theoremstyle{definition}
\newtheorem{thm}{Theorem}
\newtheorem{lem}{Lemma}
\newtheorem{defn}{Definition}
\newtheorem*{rem}{Remark}

\begin{document}
\section*{Lecture 1}
\textbf{Remarks:} Basic ordering properties are assumed to be true.
\begin{defn}
    Boundedness: The subset $A \in R$ is said to be bounded above if $\exists M$ such that $M > x~~ \forall x \in A$. And it is bounded below if $\exists m$ such that $m < x~~ \forall x \in A$. If $A$ has both then it is called bounded.   
\end{defn}
\begin{defn}
    Least Upper Bound (lub) Axiom: If $A$ is nonempty subset of $R$ and it is bounded above, then $A$ has a least upper bound in $R$.
\end{defn}
\begin{thm}
    If $A$ is nonempty subset in $R$ and it is bounded below, then it has a greatest lower bound in $R$.
\end{thm}
\begin{proof}
    We first create a set $T$ of lower bounds of $A$ $$ T = \{m \mid m < x~~\forall x \in A\}$$ $T$ is non-empty since $A$ is bounded below. Now, we need to prove that there exits a supremum of $T$ which is also a lower bound of $A$.\\
    Since, set $T$ is bounded above by all the elements of set $A$, it should have a least upper bound, say $M$ such that $ M > m~~ \forall m \in T$. Also, every element of $A$ is an upper bound of $T$ hence by definition of supremum, we can say $M \leq x~~ \forall x \in A$ hence $M$ is the lower bound of $A$. This makes it the greatest lower bound.
\end{proof}
\begin{lem}
    Suppose $A \neq \phi$ and s = lub(A) then for any $y \in A$ such that $y < s$, $\exists a \in A$ such that $ y < a \leq s$.
\end{lem}
\begin{proof}
    Suppose for contradiction, $\nexists$ any element $a$ such that $ y < a$. This means that $y \geq a, \forall a \in A$ $\implies$ $y$ is upper bound of set $A$. But $y$ is already less than least upper bound of set $A$. Hence contradiction. \\
    Therefore, $\exists$ $a \in A$ such that $ y < a \leq s$.   
\end{proof} 
\begin{thm}
    Archimedean Property: Given any positive real numbers $x,y~\exists~n \in N$ such that $ nx > y$.
\end{thm}
\begin{proof}
    Let a set $A = \{nx \mid n \in \mathbb{N}\}$. Suppose for contradiction $ nx \leq y$. Then $y$ is the upper bound of the set $A$. \\
    Let a $x > 0$, then $ y - x < y$ hence $y - x$ is not the upper bound of the set $A$. This means that $\exists m \in \mathbb{N}$ such that $y - x < mx$ $\implies$ $y < mx + x$ $\implies$ $y < (m+1)x$ which is impossible since $(m+1)x \in A$ and $y$ is upper bound of the A.\\
    $ nx > y$ is true.
\end{proof}
\begin{thm}
    If $A$ and $B$ are the two non empty bounded subsets of $R$, such that $x \leq y ~~ \forall x \in A$ and $\forall y \in B$ then $sup(A) \leq inf(B)$
\end{thm}
\begin{proof}
    Let $a$ be the supremum of $A$ and $b$ be the infimum of $B$. Therefore, \linebreak $a \geq x~~\forall x \in A$ and $ b \leq y~~ \forall y \in B$. Also, $A$ is bounded above by $B$ and elements of $B$ are the upper bound for $A$. Hence, $a \leq y ~~ \forall y \in B$. This means that $a$ is the lower bound of $B$ and $a$ is $sup(A)$. In other words, $sup(A) \leq inf(B)$. 
\end{proof}
\begin{thm}
    Given any two real number $a,b$ with $a<b$, $\exists$ $\mathbb{Q}$ between $a$ and $b$.
\end{thm}
\begin{proof}
    Since $b-a > 0$. Take two positive number $b-a$ and $1$ $\exists$ n $\in \mathbb{Z}$ such that $n(b-a) > 1$.\\ TBD
\end{proof}
\begin{thm}
    Any monotone increasing sequence of real numbers that is bounded above converges to some real number.
\end{thm}
\begin{proof}
    Let $x_n$ be a monotone increasing sequence in $\mathbb{R}$ that is bounded above by $s$ i.e. $s = lub\{x_n \mid n \in \mathbb{N}\}$ \\
    Suppose $\epsilon > 0$ $\implies s - \epsilon < s$ and $s - \epsilon$ is not the upper bound of the $x_n$.\\ 
    Using lemma 1, we can say that $\exists x_\epsilon \in x_n$ such that $s - \epsilon < x_\epsilon < s$.\\
    Using monotone condition, for some $n_0 \in \mathbb{N}$, we have,
    $$ s - \epsilon < x_n < s < s + \epsilon~~~~\forall n > n_0 \in \mathbb{N}$$ Hence $ |x_n - s| < \epsilon$. $x_n$ converges to $s~~\forall n > n_0.$ 
\end{proof}
\begin{rem}
    Nested Interval theorem $\approx$ lub $\approx$ theorem $4$
\end{rem}
\begin{proof}
    TBD
\end{proof}
\begin{thm}
    Nested Interval theorem: Suppose $\{I_n\}$ is the sequence of closed and bounded non-empty intervals such that $I_1 \supset I_2 \supset I_3$ \dots then:
    \begin{enumerate}
        \item $\bigcap\limits_{n \geq 1} I_n \neq \phi$.
        \item If the sequence of the length of the intervals goes to $0$ then $\bigcap\limits_{n \geq 1} I_n = \{x\}$. 
    \end{enumerate}
\end{thm}

\begin{proof}
    Let $I_n$ be an interval $[a_n, b_n]$ with $a_m < b_n \forall m,n \in \mathbb{N}$ . Then $\forall n \in \mathbb{N}$, $a_n$ is the increasing sequence and $b_n$ is the decreasing sequence. $b_n$ is upper bound of $a_n$ hence, $a_n < inf(b_n)$.\\
    For $b_n$, $a_n$ is the lower bound of $b_n$ i.e. $sup(a_n)< b_n$. If we combine all inequalities, we get 
    $$ a_n \leq sup(a_n) \leq inf(b_n) \leq b_n ~~ \forall n \in \mathbb{N}$$
    Using density theorem, we can say that $\exists$ some $\mathbb{Q}$ between $sup(a_n)$ and $inf(b_n)$.\\
    Hence, $\bigcap\limits_{n \geq 1} I_n \neq \phi$.\\
    Let the lenght of the interval to be $L = |b_n - a_n|$. Suppose for contradiction, we have two elements in $\bigcap\limits_{n\geq 1}I_n$ instead of one, say $x$ and $y$. \\ The distance between $x$ and $y$ is $|y - x|$. Since, $L \to 0$ hence $\exists n \in \mathbb{N}$ such that for some $n_0 \geq n$, $|L| = |b_{n_0} - a_{n_0}|< \epsilon$ for some $\epsilon > 0$. Since $|L| \to 0$, we can choose $\epsilon$ such that it is smaller than $|y - x|$. Then, if interval contains any one of the point, it can not contain the other.
\end{proof}

\hrule
\section*{Lecture 2}
\begin{defn}
    Decimals representation of Real numbers: Let $z \in \mathbb{R}^+$ be given. Let $n_0$ be the largest integer such that $n_0 \leq z$. Let $n_1$ be the largest integer such that $n_0 + \frac{n_1}{10} \leq z$. As such, say $n_k$ is defined for some $k$. Let $n_{k+1}$ be the largest integer such that $n_0 + \frac{n_1}{10^1} + \frac{n_1}{10^2} + \dots + \frac{n_k}{10^k}+\frac{n_{k+1}}{10^{k+1}} \leq z$. Consider the set of all such finite sums, i.e. the set of all
    $$ z_k = n_0 + \frac{n_1}{10^1} + \frac{n_1}{10^2} + \dots + \frac{n_k}{10^k}+\frac{n_{k+1}}{10^{k+1}} \leq z$$
    This set has a supremum and that is $z$ itself. We symbolically write $z = n_0.n_1n_2 \dots $
\end{defn}
\begin{lem}
    Let $p$ be an integer $\geq 2$. If $0 \leq a_n \leq p-1$, where $a_n$ is an integer then $\sum\limits_{n=0}^{\infty} \frac{a_n}{p_n}$ converges to some $x$ in $[0,1]$.
\end{lem}
\begin{proof}
    Since $0 \leq a_n \leq p-1$, we can replace all $a_n$ with $p-1$ and then we have$$ \sum\limits_{n=1}^{\infty} \frac{a_n}{p_n} \leq (p-1)\sum\limits_{n=1}^{\infty} \frac{1}{p_n} = 1$$
    Therefore, the sequence is bounded, and it is monotonic increasing this means it converges to some $x \leq 1$ and $x$ is already positive. Hence it converges tp some $x$ in $[0,1]$. 
\end{proof}
\begin{lem}
    Conversely, given any  $ 0 \leq x \leq 1, \exists~ a_n \in \mathbb{Z}$ and $0 \leq  a_n \leq p-1$ such that $x = \sum\limits_{n=0}^{\infty} \frac{a_n}{p_n}$.
\end{lem}
\begin{proof}
    Suppose we have $0 < x \leq 1$ and $a_1$ is the largest integer such that $ \frac{a_1}{p} < x \leq 1$. Since $x$ is bounded above by $1$, we have $ a_1 < p$ $\implies$ $a_1 \leq p-1$ since $a_1$ is an integer. Similarly, find $a_2$ such that $\frac{a_1}{p} + \frac{a_2}{p^2} < x$.This can be achieved by Archimedean property. Also, note that $a_2 \leq p-1$, since we have 
    \begin{align*}
        \frac{a_1}{p} + \frac{a_2}{p^2} &< x < 1\\
        \frac{a_1}{p} + \frac{a_2}{p^2} \leq \frac{p-1}{p} + \frac{a_2}{p^2} &< 1~~~~ (a_1 \leq p-1)\\
        1 - \frac{1}{p} + \frac{a_2}{p^2} &< 1\\
        \frac{a_2}{p^2} &< \frac{1}{p}\\
        a_2 &< p \\
        a_2 &\leq p-1
    \end{align*}
Inductively, we can define $a_n$ as the largest integer with $a_n \leq p-1$ such that $\sum\limits_{i=1}^{n} \frac{a_i}{p_i} < x$. Since $a_n < p$
TBD
\end{proof}
Suppose $\{a_n\}$ is the bounded sequence in $\mathbb{R}$, we define two sets: 
\begin{align*}
    s_n &:= inf\{a_n, a_{n+1}, \dots \}\\
    S_n &:= sup\{a_n, a_{n+1}, \dots \}
\end{align*}
Notice that $ \underset{k}{inf}(\{a_n\}) \leq s_n \leq S_n \leq \underset{k}{sup}(\{s_n\})$
\begin{defn}
    Lim superior and limit inferior: Let $a_n$ be the bounded sequence of real numbers then
    \begin{align*}
        \lim inf(a_n) &= \lim_{n\to \infty} s_n\\
        \lim sup(a_n) &= \lim_{n\to \infty} S_n
    \end{align*}
\end{defn}
Limit superior is the supremum of all subsequential limits of $\{a_n\}$. Similarly, limit inferior is the infimum of all subsequential limits of $\{a_n\}$.

Note that $s_n$ is the increasing sequence and $S_n$ is the decreasing sequence and they are bounded on both sides. Hence, we can also say that using monotone convergence theorem
\begin{align*}
    \lim inf(a_n) &= \lim_{n\to \infty} s_n = sup(s_n)\\
    \lim sup(a_n) &= \lim_{n\to \infty} S_n = inf(S_n)
\end{align*}
\begin{thm}
    A sequence $\{a_n\}$ is bounded above iff limsup $a_n < \infty$ (is finite).
\end{thm}
\begin{proof}
    TBD
\end{proof}
\begin{thm}
    A sequence $\{a_n\}$ is bounded below iff liminf $a_n < \infty$ (is finite).
\end{thm}
\begin{proof}
    TBD
\end{proof}
\begin{thm}
    Given any sequence $\{a_n\}$ there exists a subsequence $\{a_{n_k}\}$ such that $a_{n_k} \to limsup(a_n)$.
\end{thm}
\begin{proof}
    TBD
\end{proof}
\begin{thm}
    Given any sequence $\{a_n\}$ there exists a subsequence $\{a_{n_k}\}$ such that $a_{n_k} \to liminf(a_n)$.
\end{thm}
\begin{proof}
    TBD
\end{proof}
\hrule
\section*{Lecture 3}
\begin{defn}
    Finite Set: A set is finite if there exists a bijection between the set and the $\{1,2, \dots, n\}$ for some $n \in \mathbb{N}$.
\end{defn}
\begin{thm}
    $\mathbb{N}$ is an infinite set.
\end{thm}
\begin{proof}
    Negation of above definition would be a set is infinite if there does not exists a bijection between $\{1,2, \dots, n\}$ and $\mathbb{N}$. Suppose a function $f(\{1,2, \dots, n\}) \to \mathbb{N}$. \\
    We can have a natural number $f(1)+f(2)+\dots+f(n) > f(i)~~\forall i \in \mathbb{N}$ that does not have a preimage in $\{1,2, \dots, n\}$ hence the map is not bijective.
\end{proof}
\begin{thm}
    A set is infinite iff there exists a bijection from $\mathbb{N}$ to the set.
\end{thm}
\begin{proof}
    
\end{proof}

\end{document}

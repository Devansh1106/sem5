\documentclass[12pt]{report}
\usepackage{fancyhdr}
\usepackage{amsmath}
\usepackage{amsthm}
\usepackage[a4paper, width=150mm,top=25mm,bottom=25mm]{geometry}
\pagestyle{fancy}
\setlength{\headheight}{14.49998pt}
\fancyhead[L]{\small\leftmark}
\fancyhead[R]{\small\rightmark}
% \theoremstyle{definition}
\newtheorem{thm}{Theorem}
\newtheorem{lem}[thm]{Lemma}
\newtheorem{defn}{Definition}
\title{
\author{Devansh Tripathi\\ Lecturer: Prof. K. Santhesh Kumar}
}

\begin{document}
\maketitle

\section*{Lecture 1}
\begin{defn}
    Abstraction: This refers to the concept of hiding the technical details of a system and show only the essential details.
\end{defn}
\begin{enumerate}
    \item Hardware Abstraction
    \begin{itemize}
        \item Hardware level:
        \item Firmware level:
    \end{itemize}
    \item System Software Abstraction:
    \begin{itemize}
        \item OS: Interacts with hardware.
        \item Device Drivers: Specialized instruction for a specific hardware.
    \end{itemize}  
    \item Application Software Abstraction
    \begin{itemize}
        \item User level: Users do not have know the logic goes in software buiding. User should just have a knowledge of using the software. 
        \item High level programming levels: Syntaxes of high level programming langauges are more closer to the human language. It makes syntax more readable by humans.
    \end{itemize}
\end{enumerate}
`ps': For showing the process id.\\
`ps -e': For showing the process id for all programs running. `f': List users also\\
`grep $<$pattern$>$ $<$file-name$>$': find $<$pattern$>$ in the file named $<$file-name$>$.\\
ps -ef | wc -l: `$\mid$': Pipe command\\
`wc': Word count; `-l': for number of lines.\\
Add `\&' in the end of the command for keeping it background. But output will be shown in the screen. If shell is closed output is lost. \\
Add `nohub' in the starting then all the output is directed to the `nohub.out' file.\\ 
\end{document}
